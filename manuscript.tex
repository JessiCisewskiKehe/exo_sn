\documentclass[11pt, oneside]{article}   
\usepackage{geometry}                		
\geometry{letterpaper}                   		
\usepackage{graphicx}				
									
\usepackage{amssymb}
\usepackage{amsmath}
\usepackage{natbib}
\bibliographystyle{abbrvnat}
\usepackage{aas_macros} % AAS journal abbreviations

\usepackage{lscape}
%\usepackage{pdflscape} % or {lscape}
\usepackage{wasysym}
\usepackage[varg]{txfonts}
\usepackage{graphicx}
\usepackage{hyperref}
\bibpunct{(}{)}{;}{a}{}{,} % to follow the A&A style
\usepackage{longtable}
\usepackage{afterpage}
\usepackage[usenames,dvipsnames]{color}
\definecolor{mygreen}{rgb}{0,0.5,0}
\definecolor{myorange}{rgb}{0.5,0.5,0}
\definecolor{myred}{rgb}{0.5,0,0}

%\newcommand{\ep}[1]{\textcolor{red}{#1}}
%\newcommand{\epp}[1]{\textcolor{blue}{#1}}

\newcommand{\fo}{f_\mathrm{o}}
\newcommand{\fu}{f_\mathrm{u}}
\newcommand{\ic}{I_\mathrm{c}}
\newcommand{\rc}{R_\mathrm{c}}
\newcommand{\uspec}{U_\mathrm{spec}}
\newcommand{\lambdac}{\lambda_\mathrm{c}}
\newcommand{\mplanet}{M_\mathrm{p}}
\newcommand{\mearth}{M_\oplus}
\newcommand{\rearth}{R_\oplus}
\newcommand{\msun}{M_\odot}
\newcommand{\rsun}{R_\odot}
\newcommand{\mstar}{M_\star}
\newcommand{\rstar}{R_\star}
\newcommand{\mjup}{M_\mathrm{jup}}
\newcommand{\rjup}{R_\mathrm{jup}}
\newcommand{\rplanet}{R_\mathrm{p}}
\newcommand{\rplanetzero}{R_\mathrm{p,0}}
\newcommand{\rhoplanet}{\rho_\mathrm{p}}
\newcommand{\de}{\mathrm{d}}
\newcommand{\teq}{T_\mathrm{eq}}
\newcommand{\kb}{k_\mathrm{b}}
\newcommand{\htwo}{\mathrm{H}_2}
\newcommand{\htwoo}{\mathrm{H}_2\mathrm{O}}
\newcommand{\chfour}{\mathrm{CH}_4}
\newcommand{\der}{\de\rplanet/\de\ln\!\lambda }
\newcommand{\chisq}{\chi^2}
\newcommand{\chisqr}{\chi^2_\mathrm{r}}
\newcommand{\teff}{T_\mathrm{eff}}
\newcommand{\logg}{\log g}
\newcommand{\feh}{[\mathrm{Fe}/\mathrm{H}]}

\def\ms{\hbox{\,m\,s$^{-1}$}}         %m.s -1
\def\cms{\hbox{\,cm\,s$^{-1}$}}       %cm.s -1
\def\m2s2{\hbox{\,m$^{2}$\,s$^{-2}$}} %m2.s -2
\def\kms{\hbox{\,km\,s$^{-1}$}}       %km.s -1
\def\vsini{\hbox{$v$\,sin\,$i$\,}}      %vsini
\def\sini{\hbox{sin\,$i$}}      %vsini
\def\Msun{\hbox{$\mathrm{M}_{\odot}$}}             %Msun
\def\Rsun{\hbox{$\mathrm{R}_{\odot}$}}
\def\Mjup{\hbox{$\mathrm{M}_{\rm Jup}$}}
\def\Rjup{\hbox{$\mathrm{R}_{\rm Jup}$}}
\def\degr{\hbox{$^\circ$}}
\def\chisq{\mbox{$\chi^2$}}
%\def\mp{$M_{\rm p}$}
%\def\rp{$R_{\rm p}$}
\def\mp{M_{\rm p}}
\def\rp{R_{\rm p}}
\def\logrhk{$\log$(R$^{\prime}_{HK}$)}


\newcommand{\jessi}[1]{{\color{Purple}[[\textbf{Jessi: }#1]]}}
\newcommand{\xavier}[1]{{\color{blue}[[\textbf{Xavier: }#1]]}}
\newcommand{\umberto}[1]{{\color{green}[[\textbf{Umberto: }#1]]}}
\newcommand{\comment}[1]{{\color{red}[[\textbf{Referee: }#1]]}}
\newcommand{\todo}[1]{{\color{cyan}[[\textbf{TODO: }#1]]}}

%\title{Measuring cross correlation function line-profile variations in radial-velocity measurements via a Skew Normal distribution}
\title{\xavier{Measuring precise radia velocities and cross-correlation function line-profile variations using a Skew Normal distribution}}
%\thanks{Based on observations collected at the La Silla Parana Observatory,
%ESO (Chile), with the HARPS spectrograph at the 3.6-m telescope.}}

%\subtitle{II. First results of the analysis of the data set}

\author{U. Simola
	    \and X. Dumusque
	    \and Jessi Cisewski-Kehe
	    }


%\author{U. Simola \inst{1,2}
%	    \thanks{\email{umberto.simola@yale.edu}}
%	    \and X. Dumusque\inst{3}
%	    \thanks{Society in Science -- Branco Weiss Fellow (url: \url{http://www.society-in-science.org})}    
%	    \and Jessi Cisewski\inst{2}
%	    }
%
%\institute{Department of Statistical Sciences, University of Padua, Padua, Italy
%	      \and Department of Statistics and Data Science, Yale University, New Haven, CT, USA
%	      \and Observatoire de Gen\`eve, Universit\'e de Gen\`eve, 51 ch. des Maillettes, CH-1290 Versoix, Switzerland 
%	      }



\begin{document}
\maketitle

%\abstract
{\bf Abstract}\\
% Context, Aims, Methods, Results, Conclu (not mandatory)
\xavier{Stellar activity is the main limitation to the detection of small-mass exoplanets using the radial-velocity (RV) technique. Stellar activity can be probed by measuring the cross-correlation function (CCF) shape variations as a function of time. Those variations are measured using the different moments of the CCF, therefore measuring with the best precision those moments is crucial} to de-correlate exoplanet signals from spurious RV signals originating from stellar activity.

We propose here to measure those moments using a Skew Normal (SN) distribution, which compared to the Normal distribution generally used, naturally provides an extra parameter to model the CCF natural asymmetry induced by convective blueshift.
%. Since the CCF presents a natural asymmetry due to convective blueshift, the SN distribution seems to provide a more realistic model to the CCF.

We analyze 5 stars with different activity levels and different signal-to-noise ratio levels. In each case, we compare the results obtained by fitting to the CCF respectively a Normal and a SN.  We also estimate rigorous errors for the different moments of the CCF using a bootstrap analysis.

The correlations between the RVs and the CCF asymmetry or RVs and the CCF width is always stronger when using the parameters derived from the SN \xavier{in the case of real observations}. Therefore the CCF asymmetry and width derived using a SN are more sensitive to stellar activity, which \xavier{allows to probe with a better precision stellar rotational periods, in addition to characterize more precisely stellar activity signals.}
\xavier{The precision on the RVs, derived using the median of the fitted SN distributions} are on average $10\%$ smaller than the RV errors calculated on the mean of the Normal. In addition, the asymmetry parameter derived from the SN is on average $15\%$ better than the one calculated on the common Bisector Inverse Slope Span (BIS SPAN).

\xavier{We strongly encourage the use of the SN distribution rather than the Normal distribution, because this allows to retrieve in one single fit the different moments of the CCF, because the derived moments probe better stellar activity signals and because the standard errors on the RV and the the asymmetry are smaller.}
% than when using a Normal distribution, leading as well to smaller standard errors associated to the radial velocity and the asymmetry parameters.
\xavier{I just left this comment here so that we can check at the end that we answered to all referee's major comments, however, I believe it is not relevant anymore with our new analysis.}\comment{You simply cannot make this claim. What you have shown is that the SN method produces errors that are 60\% larger than the Normal distribution and you hint that this is due to it being more sensitive to activity. You have not demonstrated this by looking at the "activity" signal that the SN isolates to see if it is actually more sensitive at finding rotation periods, activity cycles, etc. Where is the sensitivity in actually getting out useful information? Furthermore, after correcting the SN RVs for the activity, you arrive at the same answer as the Normal distribution method. I would conclude that the SN offers no real advantages over Normal so the community can continue to use the standard method.
Grammatically this is an ambiguous statement. Are you looking for "better signals", i.e. ones that you like, or probing stellar signals in a better way? (This is what you meant, but not what was stated.)}
%\keywords{techniques: radial velocities -- planetary systems -- stars: activity -- methods: data analysis}


\section{Introduction}
\label{intro}
\jessi{I added subsections here for our reference to make sure the bases are covered, but we should remove them before submission.}


%%%%%%%%%%%%%%%%%%%%%%%%%%%%%%%%%%
\subsection{Goal of RV analysis}

\xavier{When working with radial velocities, the main limitation to the detection of small-mass exoplanets is not anymore the precision of the instruments used, but the different noises induce by the stars we are observing \citep[][]{Dumusque:2017aa}. Indeed, stellar oscillations, granulation phenomena and stellar activity all induce RV signals \citep[e.g.][]{Saar-1997b, Queloz-2001, Desort-2007, Dumusque-2011a, Dumusque-2016a} that are beyond the meter-per-second precision reached by the best high-resolution spectrographs. It is therefore mandatory to understand better stellar signals and to find ways to correct for them, if in the near future we want to detect or confirm an Earth-twin planet using the RV technique. This is even more true now that instrument like ESPRESSO \citep{Pepe-2014} and EXPRESS \citep{Fischer:2017aa} should have the stability to detect such signals. However, if solutions are not found to mitigate the impact of stellar activity, the detection or confirmation of potential Earth-twins will be extremely challenging.}

%for detecting Extrasolar planets and when using data from stabilized spectrographs, the different moments of the cross-correlation function (CCF) are used to measure the radial velocity of the star but also variations in the shape of the CCF \citep{Baranne-1996, Pepe-2002a}. As a planet induces a pure Doppler shift of its host star, the shape of the CCF should not vary with time. If it does, it is either due to instrumental systematics, or due to stellar signals \citep{Saar-1997b, Queloz-2001, Desort-2007, Dumusque-2014b}. On extremely stabilized spectrographs like HARPS and HARPS-N, and soon ESPRESSO, the variations observed in the shape of the CCF or mainly due to stellar activity. Therefore, measuring with the best precision all the different moments of the CCF is crucial to de-correlate exoplanet signals from spurious RV signals originating from stellar activity.}

%The radial velocity (RV) of a star is defined to be the velocity of the center of mass of the star along our line of sight. 
%\comment{No, the radial velocity you measure of the star has many components that contribute to the RV: the space motion (present even if there is no companion), the motion about the center of mass (the so-called ?-velocity), oscillations, convective blue shift, etc.
%You are just assuming that you are measuring the reflex motion of a star-planet system. There are many physical processes that produce a Doppler shift besides exoplanets.}
%This quantity can be derived precisely by measuring the Doppler shift of spectral lines produced in stellar atmospheres. 
%\comment{Well, not if you have a poor measurement precision. You only get a "precise" measurement with simultaneous wavelength calibration. Again, the authors are not being very "precise" here.}
%For spectrographs that are not stabilised in temperature and pressure, the iodine technique is used, where the light of the star passes through a iodine cell before getting into the spectrograph to imprint the absorption spectrum of iodine on top of the stellar spectrum (The Hamilton spectrograph \citep{Vogt1987} at Lick Observatory, HIRES \citep{Vogt1994} on the Keck 10-m telescope, the Tull spectrograph  \citep{Tull1995}, the High Resolution Spectrograph HRS  \citep{Tull1998}). 
%\jessi{we should mention the new spectrographs like EXPRES}
%In this case, if the spectrograph shifts due to changing atmospheric conditions, the iodine and stellar spectra are shifted in the same way. This leads to some complications when reducing the data because one has to decorrelate the iodine spectrum from the stellar spectrum. 
%\comment{This is a poorly expressed thought that is technically wrong. The spectrograph does not shift due to changing atmospheric conditions (earth's atmosphere?). It changes because of mechanical shifts (vibrations, temperature changes in spectrograph housing, changes in instrumental profile) as well as movement of the image on the slit or fiber which yes, does affect stellar and iodine lines in the same way.}  
%\comment{No, you do not decorrelate (and I am not sure what the authors mean by this word) the iodine from the stellar spectrum. The authors show that they have no understanding how the method works. What is actually done is that a high-resolution iodine spectrum is combined with a spectrum of the star without iodine lines and a fit is made to the observed star+iodine spectrum. It is not decorrelation, but rather $\chi^2$ fitting.}



%%%%%%%%%%%%%%%%%%%%%%%%%%%%%%%%%%
\subsection{Stellar activity effect on CCF, Normal fit plus FWHM and BIS SPAN indicators}
\jessi{Eventually three methods are introduced for detecting CCF asymmetry:  FWHM, bisector span, and the Figueira approach - it seems that these methods need to be better explained.  Somewhere in the intro it needs to become clear that we are (i) providing a new way of measuring CCF asymmetry, (ii) this is a unified approach that also provides an RV estimate, and (iii) it (hopefully) provides a better indicator.}

\xavier{Among the different stellar signals we are aware of, the one that is the most difficult to characterize and to correct for is the signal induced by stellar activity. Stellar activity is responsible for creating magnetic regions on the surface of stars, and those regions change locally the temperature and the convection, inducing spurious RV variations \citep[e.g.][]{Meunier-2010a, Dumusque-2014b}. In theory, it should be easy to differentiate between the pure Doppler-shift induced by a planet, that will shift the entire stellar spectrum, and stellar activity that modifies the shape of spectral lines and by doing so create a spurious shift of the stellar spectrum \citep{Saar-1997b,Hatzes-2002,Kurster2003,Lindegren-2003,Desort-2007,Lagrange-2010,Meunier-2010a,Dumusque-2014b}. However, on quiet GKM dwarfs, the main target for precise RV measurements, stellar activity induce signals of a few \ms. This corresponds physically to variations smaller than 1/100th of a pixel on the detector.
% and at this level it is extremely difficult to differentiate between a pure shift and a change in line shape. 
To be able to measure such tiny variations, we average the information of all the line in the spectrum by cross correlating the stellar spectrum with a synthetic \citep[][]{Baranne-1996,Pepe-2002a} or an observed stellar template \citep[][]{Anglada-Escude-2012}. 
%Although this process allows to measure spectral shifts at the level smaller than the meter-per-second, by averaging all the lines together we also average out line shape variation which make it more difficult to differentiate between a pure Dopler-shift and line shape variation.
The result of this operation gives us the cross-correlation function (CCF). To measure the Doppler-shift between different spectra, and therefore get the RV of a star as a function of time, we calculate the variations of the CCF barycenter. The barycenter is generally estimated by fitting a Normal distribution to the CCF, and taking its mean. Variation in line shape between different spectra, which indicate the presence of signals induced by stellar activity, is measured by analysing the different moments of the CCF. Generally, the width of the CCF is estimated using the FWHM of the fitted Normal distribution, and its asymmetry using the the bisector inverse slope span \citep[BIS SPAN,][]{Queloz-2001}.
%bisector \citep[][]{Voigt1956}, the bisector curvature \citep[][]{Hatzes1996} or the bisector inverse slope span \citep[BIS SPAN,][]{Queloz-2001}.
}

\subsection{Why FWHM and BIS SPAN important?}

\xavier{If a RV signal is induced by activity, generally a strong correlation will be observed between the RV and chromospheric activity indicators like \logrhk\,or H-$\alpha$ \citep[e.g.][]{Boisse-2009,Dumusque-2012,Robertson-2014}, but also between the RV and the FWHM of the CCF or its BIS SPAN \citep[][]{Queloz-2001,Boisse-2009,Queloz-2009,Dumusque-2016a}. It is therefore common now, that when fitting for a planetray signal, in addition to a Keplerian, the model includes in addition linear dependancies with the \logrhk, the FWHM and the BIS SPAN \citep[e.g.][]{Dumusque:2017aa,Feng:2017aa}. It is also common to add a Gaussian process to the model to account for the correlated noise induced by stellar activity. The hyperparameters of the Gaussian process are generally trained on the different activity indicators \citep[e.g.][]{Haywood-2014,Rajpaul-2015}. It is therefore essential for mitigating stellar activity to obtain activity indicators that are the most correlated to the RVs but also for which we can obtain the best precision.}

\subsection{Figueira indicators of stellar activity + other}
\xavier{Several works have derived indicators that are more sensitive to line asymmetry than the BIS SPAN. In \citet{Boisse-2011}, the authors develop $V_{span}$ a new indicator to derive the CCF asymmetry, that is more sensitive than the BIS SPAN at low signal-to-noise ratio (SNR). \citet{Figueira-2013} studied the use of two new indicators, bi-Gauss and $V_{asy}$. The authors were able to show that when using bi-Gauss, the amplitude in asymmetry is 30\% larger, therefore allowing the detection of smaller-amplitude correlations with radial-velocity variations. They also demonstrated that $V_{asy}$ seems to be the best indicator of line asymmetry at high SNR, as its correlation with RV is more significant than any other asymmetry indictors.
}

\subsection{Why using a SN distribution?}
\xavier{In all the method described above, the RV and the FWHM is derived using a Normal distribution fitted to the CCF, and the asymmetry is estimated using another approach. In this paper we propose to use a Skew Normal (SN) distribution to derive at the same time the RV, the FWHM and the asymmetry of the CCF, as this function includes a skewness parameter \citep[][]{Azzalini1985}. In addition, we know that for solar-type stars and cooler dwarfs, the bisector of the CCF as a "C"-shape due to convective blueshift \citep[e.g.][]{Dravins-1981, Gray-2009}. Therefore, fitting the CCF using a model that naturally includes an asymmetry, like the SN distribution, should give more precise results.
}



%For spectrographs that are stabilized, the spectrum of a calibration lamp is recorded close to the stellar spectrum on the CCD, which prevents contamination of the stellar spectrum (CORALIE \citep{Queloz-2000a}, The High Accuracy Radial Velocity Planet Searcher (HARPS) \citep{Mayor-2003}, HARPS- N \citep{Cosentino-2012}, SOPHIE \citep{Bouchy:2013aa}, CARMENES \citep{Quirrenbach:2014aa}). For those instruments, reducing the data is easier as the stellar spectrum is not contaminated with iodine absorption lines.
%\comment{This is not true! I have taken spectra with HARPS and there is cross-talk between spectra and Th-Ar particularly for strong emission lines. So there is some contamination. Furthermore, this "preventing of contamination" as the authors state has nothing to do with the spectrograph being stablized, it has to do with the simultaneous wavelength calibration using two fibers. In fact one reference the authors give, CORALIE, is not a stabilized spectrograph!}
%\comment{In paragraph 1 you give two lists of spectrographs. I presume the first one is a list spectrographs that use the iodine method, although that is not explicitly expressed. The authors talk about the iodine cell and then just give a parenthetical list of spectrographs. There is no explicit statement that these spectrographs, designed for other purposes, are all equipped with an iodine cell. The reader must assume this. Furthermore, the list is not all-inclusive. I can think of a dozen facilities that use iodine cells. So you should say, "For example" or..."to name a few"
%The second list it is for simultaneous Th-Ar calibration. Not all of these are "stabilized" like the authors claim. CORALIE, if I am not mistaken, was simultaneous Th-Ar without stabilization. That is why they built HARPS. Furthermore, all of theses spectrographs were designed for precise stellar RVs, unlike the spectrographs listed for the iodine method.}

%For stabilized spectrographs, the RV is derived by first correlating a stellar spectrum at a particular time with a synthetic \citep[][]{Baranne-1996,Pepe-2002a} or an observed stellar template \citep[][]{Anglada-Escude-2012}, which gives an average line profile, generally called Cross Correlation Function (CCF).  The CCF is a function of RV, and the RV corresponding to the minimum of the CCF is the estimated RV for that spectrum.  However, this would give a noisy estimate so first a Normal distribution shape is fitted to the CCF, and then the estimated mean of the Normal distribution is used as the estimate of the RV.  The full width at half maximum (FWHM) of the Normal fit is also used as an indicator of spectral line asymmetry.
%The CCF technique allows for an averaging of thousands of spectral lines and can therefore reach a very high signal-to-noise ratio (SNR), which is essential for a good RV precision.

%The convection in external layers of solar type stars is responsible for the granulation pattern than can be seen at high spatial resolution on the surface of the Sun. The differences in flux and velocity between upflows and downflows change the Normal profile of spectral lines that become asymmetric with a "C"-shaped profile \citep[][]{Dravins-1981}. 
%\comment{Regarding the C-shape of the bisector. The classic C-shape is only for late-type stars. In fact for hotter stars (F-type) it reverses.
%The strength of the asymmetry depends not just on the velocities, it is more complicated than that. It also depends on the flux ratio between hot cells and cool lanes, as well as the ratio of surface areas between the two.}
%The strength of the asymmetry depends on the velocity of the convection, approximatively 300\ms for the Sun, but also on the formation depth of spectral lines \citep{Gray-2009}. Since the CCF is an average of all the spectral lines, some strongly asymmetric and some not, its asymmetry is rather small, which is why a Normal distribution can be an appropriate model for the CCF. This small asymmetry modifies however slightly the estimated RVs of the star, reducing the accuracy of the measurement, but if this asymmetry does not vary with time, the precision is kept.
%\comment{But the activity causes the asymmetry to change due to rotation, spot evolution, etc. The authors are confusing the readers here. They talk about a changing activity signal, but now argue that the precision is only kept if the asymmetry is constant, which it is not. I think what they want to say is that you measure a precise stellar radial velocity, but the activity makes its own contribution which reduces the accuracy of the RV determination for the barycentric motion due to a planet.}
%
%
%Convection is not the only phenomenon responsible for asymmetries in the single spectral line and the CCF. Stellar activity is responsible for the appearance of dark spots and bright faculae on the stellar photosphere, which breaks the flux balance between the red-shifted and the blue-shifted halves of a rotating star and therefore induce an asymmetry of spectral lines and thus of the CCF. As the star rotates, spots and faculae move across the stellar disk, modifying the line asymmetry and thus producing an apparent Doppler shift \citep{Saar-1997b,Hatzes-2002,Kurster2003,Desort-2007,Lagrange-2010,Boisse-2012b}. Spots and faculae are also regions where the magnetic field is strong. Strong magnetic fields reduce stellar convection, which in turn modifies the asymmetry of spectral lines \citep[][]{Cavallini-1985a,Dravins-1981,Lindegren-2003,Meunier-2010a,Dumusque-2014b}.
%
%%%%%%%%%%%%%%%%%%%%%%%%%%%%%%%%%%%
%\subsection{Bisector span indicators of stellar activity}
%\jessi{Its seems that more details should be added here}
%Stellar activity can induce RV variations by a modification of the spectral line asymmetry, while any orbiting companions would induce a pure Doppler shift on \emph{all} spectral lines without modifying their shape.
%Therefore, assuming that there is no instrumental systematics, stellar activity generally induces a variation in line asymmetry that can be observed in the  FWHM of the CCF. 
%The line asymmetry is commonly retrieved by calculating the bisector of the CCF \citep[][]{Voigt1956} and deriving the bisector curvature \citep[][]{Hatzes1996} or the bisector inverse slope span \citep[BIS SPAN,][]{Queloz-2001}. 
%An orbiting exoplanet will produce an RV variation induced by a pure Doppler shift of \emph{all} the spectral lines of each stellar spectrum. 
%Stellar activity, however, does not produce a blueshift or redshift of the spectra, but can create a spurious RV signal by modifying the shape of spectral lines. To track changes in line shape, the FWHM, the BIS SPAN or the indicators introduced by \citet{Figueira-2013} are often used, which provide information on the average width and asymmetry of the CCF \citep{Hatzes1996, fiorenzano2005line, Queloz-2001}. A strong correlation between the estimated RVs and one or more of these parameters is a sign that the estimated RVs may be contaminated by stellar activity. 
%\comment{It is the slope or span, not both. And you generally do not calculate the inverse, just the
%span. This may be correlated or inversely correlated with the RV.}
%
%
%%%%%%%%%%%%%%%%%%%%%%%%%%%%%%%%%%%
%\subsection{Figueira indicators of stellar activity}
%\citet{Figueira-2013} proposed different indicators, including a bi-Gaussian fitting of the CCF \jessi{this needs to be better explained}. 


%%%%%%%%%%%%%%%%%%%%%%%%%%%%%%%%%%
%\subsection{Issues with detecting stellar activity}
%Unfortunately, when analyzing slow rotators stars such as the Sun, due to the limited spectral resolution of the spectrographs and the limited precision in RV, it becomes difficult to measure the line asymmetry, resulting in complications for detecting very small-mass planets with the RV technique.
%In the procedure described above, the measurement of the RV and the FWHM is done separately from the measurement of the line asymmetry. All those parameters are correlated when stellar activity is dominant, and performing a step-by-step approach makes it difficult to correctly derive the errors on the different parameters retrieved. 
%\comment{I simply do not understand this statement, possibly because it does not follow a standard English construction. You have a CCF and you measure a width and bisector span. This is a measurement error determined by the S/N, the resolution, and the stability of the spectrograph. You can determine an error in a quantity (RV and BIS). You can also have an uncertainty in the contribution of activity to the RV signal, but that is not a formal error. True errors come from photon statistics, systematic errors, instrumental errors, etc. You can attempt to remove the signal due to activity and that has an associated error, but that error is due to your lack of knowledge of the surface structure causing this RV and the exact value of the contribution to the RV. This is different from a measurement error. Maybe that is what you mean here, but it is not clear from what is written.}
%In addition, the Normal distribution cannot take into account the natural asymmetry of the CCF, leaving correlated noise in the residuals, which also complicates the determination of errors. We propose to overcome these problems by fitting a SN (SN) distribution to the CCF, which naturally includes a skewness parameter \citep[][]{Azzalini1985}.


%%%%%%%%%%%%%%%%%%%%%%%%%%%%%%%%%%
\subsection{Outline of paper}
\todo{This paragraph will need to be updated once the rest is finished.} The paper is organized as follow. In Sec.~\ref{sec:2} we introduce the SN distribution and describe its applicability for modeling the CCF,  show that the SN distribution is a better representation of observed CCF than a Normal distribution, and study how the SN parameters relate to the RV, FWHM and BIS SPAN of the CCF. In Sec.~\ref{intro} we present a simple model to correct for stellar activity. In Sec.~\ref{sec:4}, we compare on real observations the sensitivity of the SN parameters to stellar activity with respect to other existing indicators. In Sec.~\ref{sec:5} we derive error bars for the different CCF parameters, and finally we discuss our results and conclude in Sec.~\ref{sec:discu} and Sec.~\ref{sec:conclu}.



%-----------------------------------------------------------------------------------------------------------------------------------------------
\section{The Skew Normal distribution} \label{sec:2}

The Skew Normal (SN) distribution is a class of probability distributions which includes the Normal distribution as a special case \citep{Azzalini1985}. The SN distribution has, in addition to a location and a scale parameter analogous to the Normal distribution's mean and standard deviation, a third parameter which describes the asymmetry, or the skewness, of the distribution. Considering a random variable $Y\in \mathbb R$ (where $\mathbb R$ is the real line) which follows a SN distribution with location parameter $\xi \in \mathbb R$, scale parameter $\omega \in \mathbb R^{+}$ (i.e., the positive real line), and skewness parameter $\alpha \in \mathbb R$, its density at some value $Y = y$ can be written as 
\begin{equation} \label{def:snd_gen}
SN(y;\xi, \omega, \alpha) = \frac{2}{\omega} \phi\left(\frac{y-\xi}{\omega}\right) \Phi\left(\frac{\alpha(y-\xi)}{\omega}\right),
\end{equation}
where $\phi$ and $\Phi$ are respectively the density function and the distribution function of a \emph{standard} Normal distribution\footnote{A standard Normal distribution is a Normal distribution with a mean of 0 and a standard deviation of 1.}
 and $\alpha \in \mathbb R$ is the skewness parameter which quantifies the asymmetry of the SN.  We then write $Y \sim SN(\xi, \omega^{2}, \alpha)$ to mean that the random variable $Y$ follows the noted SN distribution.
Examples of SN densities under different skewness parameter values and the same location and scale parameters ($\xi = 0$ and $\omega = 1$) are displayed in Fig.~\ref{fig:SN.plot}.  A usual Normal distribution is the special case of the SN distribution when the skewness parameter, $\alpha$, is equal to 0.\footnote{This can be seen from Eq.~\eqref{def:snd_gen}.  If $\alpha = 0$ then $\Phi\left(\frac{\alpha(y-\xi)}{\omega}\right) = \Phi(0)$; this is the the probability a standard Normal random variable is less than or equal to 0, which is $0.5$.  The $0.5$ cancels with the $2$ in the denominator and what remains is the usual Normal density, $\frac{1}{\omega} \phi\left(\frac{y-\xi}{\omega}\right)$}
%
\begin{figure}[htbp]
   \centering
\includegraphics[height = 3in]{Skew_Normal_densities_jjck.pdf} 
   \caption{Density function of a random variable $Y \sim SN(\xi, \omega^{2}, \alpha)$ with location parameter $\xi = 0$, scale parameter $\omega = 1$ and different values of the skewness parameter $\alpha$ indicated by different colors and line types. Note that the solid black line has an $\alpha = 0$, making it a Normal distribution.}
   \label{fig:SN.plot}
\end{figure}
%
For reasons related to the interpretation of the parameters in Equation~\eqref{def:snd_gen} and computational issues with estimating $\alpha$ near 0, a different parametrization is used, which is referred to as the \emph{centered parametrization} (CP).  We will be using the CP in this work, which includes a mean parameter $\mu$, a variance parameter $\sigma^2$ and a skewness parameter $\gamma$.  In order to define the CP, we need to express the CP parameters $(\mu, \sigma^2, \gamma)$ as a function of the one used in the Equation~\eqref{def:snd_gen} with $(\xi, \omega^2, \alpha)$ by
%
\begin{equation} \label{eq:snd_cp}
\mu = \xi + \omega \beta, \quad \sigma^{2} = \omega^{2}(1-\beta^2), \quad \gamma = \frac{1}{2}(4-\pi) \beta^{3}\left(1-\beta^2\right)^{-3/2},
\end{equation}
%
where $\beta = \sqrt{\frac{2}{\pi}}\left(\frac{\alpha}{\sqrt{1+\alpha^2}}\right)$.

By using Eq.~\eqref{eq:snd_cp}, the new set of parameters $(\mu, \sigma^2, \gamma)$ provides a more clear interpretation of the behavior of the SN distribution. For the $\alpha$ values used in Fig.~\ref{fig:SN.plot}, the corresponding values of $\mu$, $\sigma^2$, $\gamma$ are displayed in Table~\ref{tab:cp_values}.  In particular, $\mu$ and $\sigma^2$ are the actual mean and variance of the distribution (rather than simply a location and scale parameter) and $\gamma$ becomes an index for evaluating the skewness of the SN. 

Beyond the mean of the SN, we introduce a second location parameter that will be largely used in the analyses: the median. The median of the SN, and in general the median of an absolute continuous random variable, is defined as that value $m$ such that 
%
\begin{equation} \label{eq:snmed}
\int_{-\infty}^{m} SN(y;\xi, \omega, \alpha) = \frac{1}{2},
\end{equation}
%
where $SN(y;\xi, \omega, \alpha)$ follows Eq.~\eqref{def:snd_gen}.\footnote{We recall that when using a symmetric distribution such as the Normal distribution, the mean and the median are equivalent.}

%% Requires the booktabs if the memoir class is not being used
\begin{table}[htbp]
   \centering
   %\topcaption{Table captions are better up top} % requires the topcapt package
   \begin{tabular}{|cccc|} % Column formatting, @{} suppresses leading/trailing space
\hline
$\alpha$ & $\mu$ & $\sigma^2$ & $\gamma$ \\
\hline
 -3 	&	 -0.757	&	 0.427	&	 -0.667 \\
0	&	 0.000 	&	1.000	&	 0.000 \\
2	&	 0.714	&	 0.491	&	 0.454\\
6	&	 0.787	&	 0.381	&	 0.891\\
10	&	 0.794	&	 0.370	&	 0.956\\
\hline
   \end{tabular}
   \caption{CP values, $(\mu, \sigma^2, \gamma)$, corresponding to the $\alpha$ values from Fig.~\ref{fig:SN.plot} (with location parameter $\xi = 0$ and scale parameter $\omega = 1$) using Eq.~\eqref{eq:snd_cp}.  Values are rounded to three decimal places.}
   \label{tab:cp_values}
\end{table}
%
Further details about the parametrization from Eq.~\eqref{def:snd_gen} (called \emph{Direct Parametrization} or DP), the CP, and general statistical properties of the SN are treated in rigorous mathematical and statistical viewpoints in the book by \cite{Azzalini2014}.

%-----------------------------------------------------------------------------------------------------------------------------------------------
\subsection{Fitting the Skew Normal distribution to the CCF} \label{sec:3}

%The CCF represents the average shape of spectral lines and is expressed in flux as a function of radial-velocity.
The SN density shape is used to model the CCF.  In particular, using least-squares algorithm, we fit the following function:
%
\begin{eqnarray} \label{eq:3}
f_{CCF}(y_i) = \mathrm{C} - \mathrm{A} \times SN(y_i;\mu, \sigma^2, \gamma), \quad i = 1, \ldots, n
\end{eqnarray}
%
where C is an unknown offset fitting the continuum of the CCF, A is an unknown amplitude parameter and $y_1, \ldots, y_n$ are the set of RV's considered for the CCF. Note that the CCF is expressed in flux as a function of the lag of the cross-correlation template, expressed in RV.

Since the CCF presents a natural asymmetry due the convective blueshift, the SN distribution should in principle better catch this aspect, together with other changes in asymmetry, respect the Normal fit. To initially check this assumption, we compared the CCF residuals after fitting a Normal and a SN distribution for 2 stars. The first star is Alpha Centauri b, whose CCFs have high signal-to-noise ratio (SNR). The second star is Corot-7, whose CCFs have low SNR. Fig.~\ref{fig:Residual.comparison} shows that the SN seems to be a slightly better model to explain the shape of the CCF, in particular as the SNR decreases.
%
\begin{figure*}[htbp]
   \centering
\includegraphics[height = 2.5in]{[1]HD128621Residuals.pdf} 
\includegraphics[height = 2.5in]{[1]LRa01_E2Residuals.pdf} 
   \caption{Comparison between the Normal (black circles) and the SN (red crosses) residuals using CCFs from the star Alpha Centauri b (left) and Corot-7 (right). When looking at the residuals corresponding to the tails of the CCF, the results derived by the two fits are comparable. However, when focusing on the center of the CCF, the SN fit leads to slightly better results for both the stars. Moreover, as the SNR decreases, the SN distribution shows smaller residuals respect the Normal ones.}
    \label{fig:Residual.comparison}
\end{figure*}
%

In the following of the paper, we define RV as the mean of the Normal distribution. Concerning the fit of the CCF using the SN, we present at first $2$ indicators that define the RV of the star: the mean of the SN, defined as RV mean SN and the median of the SN, defined as RV median SN (i.e. realling Eq.\eqref{eq:snmed}, $m$ = RV median SN). We will discuss advantages and limits for both these choices in Section \ref{sec:4} and Section \ref{sec:5}. For the width of the CCF, we use the FWHM of the Normal, which is $2\sqrt(2\ln2)\sigma$. The width of the SN, SN FWHM, is defined in the same way\footnote{Note that SN FWHM does not correspond to the width of the SN distribution at half maximum like in the Normal case.}. Being a Normal distribution symmetric, there is not such a parameter that evaluates the asymmetry of the distribution, so the BIS SPAN is used. The BIS SPAN will be compared to the asymmetric parameter $\gamma$ of the SN, named SN GAMMA. To test the strength of the correlation between the RVs and the different indicators for stellar activity, we calculated the Pearson correlation coefficient, $R$.
A $p-$value for the statistical test having null hypothesis $H_{0}: R=0$ (i.e., no correlation) is provided, along with a $95\,\%$ confidence interval for $R$.

%-----------------------------------------------------------------------------------------------------------------------------------------------
\section{Simulation Study} \label{sec:soap}
In order to evaluate the performance of the proposed SN approach for modeling the CCF, we begin by considering a simulation study using spectra generated from the Spot Oscillation And Planet (SOAP) 2.0 code \citep{Dumusque-2014b}.

\umberto{Add more information on SOAP, the way we retrieve the CCFs with SOAP, the characteristics of the CCF (i.e. how many points, ...) and the reasons why we need CCFs simulated with SOAP.}
\umberto{Specify characteristic of the star.}

\umberto{Add info on the standard errors about the asymmetry parameter and the width and make a comparison with the results from the bootstrap. Then right discussions and conclusions,}

\subsection{Faculae} \label{sec:soap.faculae}

\umberto{Specify characteristic of the faculae.}

Fig.~\ref{fig:faculae} shows the results obtained when a faculae is present on the photosphere of the star. It is possible to note that, although there is not a planet, the presence of the faculae leads to spurious variations in RVs for all the proposed indicators. SN mean RV seems to have the smallest spurious variations caused by the faculae. %Anyway, the evaluation of the standard errors retrieved by using a bootstrap analysis that will be detailed discussed in Section \ref{sec:5}, shows that RV mean SN has the largest related uncertainties. The standard errors associates with RV median SN and RV are comparable, although the standard errors retrieved when the RV is defined as the median of the SN are $10 \%$ smaller than the standard errors derived for RV.

Since in this case we know that the variations in RV are only caused by the faculae, the evaluation of the correlations between the set of RVs and the asymmetry parameters can provide relevant information to understand the cause of the variations observed in RVs. Fig.~\ref{fig:faculae.corr} shows the correlations obtained using as RV respectively RV mean SN, RV median SN and RV. The $\gamma$ parameter shows a correlation with RV median SN equal to $-0.944$, which is stronger than the correlations between the other asymmetry indicators and their estimated RVs. The correlation between the width of the CCF and RV mean SN is the strongest one ($R^{2}=0.655$).

\begin{figure*}[htbp]
   \centering
\includegraphics[height = 4in]{RV_comparison_FACULAE.pdf} 
%\includegraphics[height = 2.5in]{RV_se_comparison_FACULAE.pdf} 
\caption{(RVs changes as function of the orbital phase in the case in which a faculae is present on the photosphere of the star. SN mean RV seems to have the smallest spurious variations caused by the faculae.}
   %\caption{(left) RVs changes as function of the orbital phase in the case in which a faculae is present on the photosphere of the star. SN mean RV seems to have the smallest spurious variations caused by the faculae. (right) Evaluation of the standard errors corresponding to the defined RVs. The standard errors retrieved for RV median SN are $10 \%$ smaller than the standard errors derived for RV. RV mean SN has the largest related uncertainties.}
    \label{fig:faculae}
\end{figure*}

\begin{figure*}[htbp]
   \centering
\includegraphics[height = 4in]{SOAP_FACULAE_Comparison_para_SN.pdf} 
   \caption{Evaluation of the correlation between the RVs and the asymmetry parameters when a faculae is present on the photosphere of the star. In this case both the shape and the width of the CCF changes as the faculae moves, producing statistically significative correlations between the RVs and respectively the asymmetry parameter and the width parameter.}
    \label{fig:faculae.corr}
\end{figure*}

\subsection{Spot} \label{sec:soap.spot}

\umberto{Specify characteristic of the spot.}

Fig.~\ref{fig:spot} shows the results obtained when a spot is present on the photosphere of the star. Similarly to the previous case, the presence of the spot leads to spurious variations in RVs for all the proposed indicators, but N mean RV seems to have the smallest ones. We note however that in this case all the three indicators are similarly effected by the spurious changes in RVs caused by the spot. %The evaluation of the standard errors, as before, shows that RV mean SN has the largest variability, while the standard errors associated with RV median SN are $10 \%$ smaller than the standard errors associated with RV. 

Fig.~\ref{fig:spot.corr} shows the correlations between the asymmetry parameters and respectively RV mean SN, RV median SN and RV. The correlation between $\gamma$ and RV median SN is equal to $-0.89$ while the correlation between $\gamma$ and RV median SN is equal to $-0.82$. Both these correlations are stronger than the correlation between the BIS SPAN and RV ($R^2=-0.77$). The correlations between the width of the CCF and the corresponding RVs are slightly stronger when fitting a SN rather than a Normal distribution, as shown in the left series of plots in Fig.~\ref{fig:spot.corr}. The results shows anyway a correlation close to $0$, suggesting that the presence of a spot on the photosphere of star changes the shape of the CCF rather than its width.

\begin{figure*}[htbp]
   \centering
\includegraphics[height = 4in]{RV_comparison_SPOT.pdf} 
%\includegraphics[height = 2.5in]{RV_se_comparison_PLANET_SPOT.pdf} 
\caption{RVs changes as function of the orbital phase in the case in which a spot is present on the photosphere of the star. SN mean RV seems to have the smallest spurious variations caused by the faculae.}
   %\caption{(left) RVs changes as function of the orbital phase in the case in which a spot is present on the photosphere of the star. SN mean RV seems to have the smallest spurious variations caused by the faculae. (right) Evaluation of the standard errors corresponding to the defined RVs. The standard errors retrieved for RV median SN are $10 \%$ smaller than the standard errors derived for RV. RV mean SN has the largest related uncertainties.}
    \label{fig:spot}
\end{figure*}

\begin{figure*}[htbp]
   \centering
\includegraphics[height = 4in]{SOAP_SPOT_Comparison_para_SN.pdf} 
   \caption{Evaluation of the correlation between the RVs and the asymmetry parameters when a spot is present on the photosphere of the star. In this case only the shape of the CCF changes as the spot moves, producing statistically significative correlations only between the RVs and the asymmetry parameter.}
    \label{fig:spot.corr}
\end{figure*}

\subsection{Spot and planet} \label{sec:soap.spot.planet}

\umberto{Specify characteristic of the spot.}

Fig.~\ref{fig:spot.plus.planet} shows the results obtained when a spot is present on the photosphere of the star and a planet is injected. The planet, having an amplitude of 10 \ms, produces a pure doppler shift on the CCF, without further changing its shape.  In this case N mean RV seems to have the largest variations caused by the combined action of spot and planet. %The evaluation of the standard errors leads to the same conclusions reported for the two previous cases: RV mean SN has the largest related uncertainties and the standard errors associates with RV median SN are $10 \%$ smaller than the standard errors derived for RV.

Fig.~\ref{fig:spot.plus.planet.corr} shows the correlations between $\gamma$ and respectively RV mean SN and RV median SN and the correlation between BIS SPAN and RV. In this case the correlations are weaker than the ones derived when only a spot is present on the photosphere of the star. Anyway the $\gamma$ parameter shows a correlation with the median SN of $-0.433$, which is stronger than what the correlation between the other asymmetry indicators and their corresponding RVs. The correlations between the width of the CCF and the corresponding RVs are comparable and, similarly to the previous case, close to $0$.

\begin{figure*}[htbp]
   \centering
\includegraphics[height = 4in]{RV_comparison_PLANET_SPOT.pdf} 
%\includegraphics[height = 2.5in]{RV_se_comparison_SPOT.pdf} 
 \caption{RVs changes as function of the orbital phase in the case in which a spot is present on the photosphere of the star and a planet is injected. N mean RV seems to have the largest variations caused by the combined action of spot and planet.}
   %\caption{(left) RVs changes as function of the orbital phase in the case in which a spot is present on the photosphere of the star and a planet is injected. N mean RV seems to have the largest variations caused by the combined action of spot and planet. (right) Evaluation of the standard errors corresponding to the defined RVs. The standard errors retrieved for RV median SN are $10 \%$ smaller than the standard errors derived for RV. RV mean SN has the largest related uncertainties.}
    \label{fig:spot.plus.planet}
\end{figure*}

\begin{figure*}[htbp]
   \centering
\includegraphics[height = 4in]{SOAP_SPOT_PLANET_Comparison_para_SN.pdf} 
   \caption{Evaluation of the correlation between the RVs and the asymmetry parameters when a spot is present on the photosphere of the star and a planet is injected.  In this case only the shape of the CCF changes as the spot moves, producing statistically significative correlations only between the RVs and the asymmetry parameter. The correlations between the RVs and the width parameter of the CCF is weaker than the previous case that considers only the presence of a spot on the photosphere of the star.}
    \label{fig:spot.plus.planet.corr}
\end{figure*}

As final step we removed to the RVs obtained in this case those spurious signals coming from the spot. The results, presented in Fig.~\ref{fig:planet-spot}, show that the periodic pure doppler shift caused by the planet is detected.

\begin{figure*}[htbp]
   \centering
\includegraphics[height = 4in]{RV_comparison_PLANET.pdf} 
%\includegraphics[height = 2.5in]{SOAP_PLANET_Comparison_para_SN.pdf} 
   \caption{RVs changes as function of the orbital phase after removed the spurious effect of the spot. In green the real signal of the planet is displayed.}
    \label{fig:planet-spot}
\end{figure*}

In this Section we presented a first implementation of the SN fit to CCF, using the simulation environment SOAP. Before moving to the next Section, where the analyses on five real stars are presented, we need to provide further considerations. First of all, looking ad the analyses conducted with SOAP, it seems that the largest correlation between an asymmetry parameter and a set of RVs happens to be when respectively $\gamma$ and RV median SN are used. This is a bit surprising, since as the shape of the CCF changes, we expect RV median SN to be more robust than RV mean SN. \umberto{A possible justification of this ...}. As second, when searching for stellar activity by deriving the correlation between the set of RVs and either an asymmetry parameter or the width of the CCF, the latter leads to weaker and hence less conclusive results if the active region is a spot. When stellar activity is dominated by faculae, both the shape and the width of the CCF changes as the faculae evolves on the photosphere of the star. Finally, the correlations involving the common indicators (i.e. RV, FWHM and BIS SPAN) are systematically weaker than the correlations obtained by fitting the SN to the CCF, suggesting that this distribution could be helpful when searching for active regions. We recall moreover that all the quantities needed for conducting the analyses of the CCF are directly available by just fitting the SN.

We remand the reader to Section \ref{sec:5} for the evaluation of the standard errors associated with the parameters estimated for all the three presented cases. This step will point out how RV median SN minimizes the uncertainties respect using both RV mean SN and RV, therefore suggesting its use in order to properly define the set of RVs of the star.

%-----------------------------------------------------------------------------------------------------------------------------------------------
\section{Real data applications} \label{sec:4}

In this Section we present the analyses conducted on Alpha Centauri b, comparing the result of fitting a CCF using the SN distribution defined in Section \ref{sec:3} with the approach based on the Normal distribution. Other four stars have been analyzed with the proposed method and details can be found in the Appendix \ref{appendix}. For all the stars that have been considered in the present work, we selected those CCFs having SNR larger than 10.

 A comparison with the results obtained by the classic approach is done, where the RVs of the star are estimated by retrieving the mean of the Normal distribution used to fit the CCF, along with the BIS SPAN or the other asymmetric parameters defined in \citet{Figueira-2013}. The latter parameters are calculated separately from the Normal fit that leads to the set of RVs of the star.

\subsection{Alpha Centauri B} \label{sec:alphacentb}

A total of $1808$ CCFs measured in $2010$ have been analysed from the star Alpha Centauri B. Several measurement in 2010 are contaminated by light from Alpha Centauri A. To remove contaminated spectra and thus CCFs, we performed the same selection as presented in \citet{Dumusque-2012}. Moreover, as noted in \citet{Dumusque-2012} and \citet{Thompson-2017}, this dataset presents a strong stellar activity signal.

We begin the analyses by evaluating the correlation between $\gamma$ and the BIS SPAN. In Fig.~\ref{fig:alphacent:corr.gamma}, we see that the relationship between $\gamma$ and the BIS SPAN is linear, with a slope equal to $0.72$ and a strong Pearson correlation coefficient of $0.954$. This comparison is useful because $\gamma$ is an adimensional parameter taking information about the asymmetry of the SN while the BIS SPAN, beyond this, has got unit of measure of \kms. In other words, by using Fig.~\ref{fig:alphacent:corr.gamma}, it is possible to provide a physical meaning to $\gamma$.
%
\begin{figure}[htbp]
   \centering
\includegraphics[height = 4in]{HD12862_[2]gamma_vs_bisspan.pdf} 
   \caption{Correlation between $\gamma$ and the BIS SPAN for Alpha Centauri B. Because $\gamma$ is adimensional, retrieving the slope between $\gamma$ and the BIS SPAN, which is expressed in \kms, allows us to provide physical meaning to $\gamma$.}
   \label{fig:alphacent:corr.gamma}
\end{figure}
%

Fig.~\ref{fig:alphacent:diff:RV} shows the comparison between the RVs retrieved using the SN shape and the ones obtained with the Normal shape. It is possible to appreciate the presence of a strong stellar activity signal, as expected \citep{Dumusque-2012,Thompson-2017}. When using RV mean SN, it is possible to observe more variations than the ones measured by the Normal fitting. This happens because the mean of the SN is more sensitive to stellar activity. In fact, because the SN includes an asymmetry parameter, RV mean SN gets more shifted in the direction of the asymmetry induced by stellar activity. On the other hand, when using RV median SN, smaller variations in RV are caused by changes in the asymmetry of the CCF, because this second location parameter is a more robust indicator than the mean. The bottom plot of Fig.~\ref{fig:alphacent:diff:RV} captures this aspect. Both indicators can be used to capture and summarise the different information available in the CCF, as will be shown in the remainder of this work.
%
\begin{figure*}[htbp]
   \centering
\includegraphics[height = 4in]{HD12862_[2]RadialVelocityDifferences.pdf} 
   \caption{(top) RVs as function of Julian Day for Alpha Centauri b. The RVs are retrieved using the mean of the Normal (red triangles), RV mean SN (black circles), RV median SN (cyan crosses). (bottom) RV differences between Normal RV and RV mean SN (black circles) and between Normal RV and a RV median SN (cyan crosses).}
   \label{fig:alphacent:diff:RV}
\end{figure*}
%

Similar to \citet{Figueira-2013}, we compare the correlation between the different activity indicators and the RVs of the star in Fig.~\ref{fig:alphacent:corrPlot}. The correlation between $\gamma$ and RV mean SN and the correlation between $\gamma$ and RV median SN are much stronger than the correlations calculated between the other asymmetry parameters and their corresponding RVs. In particular the correlation between $\gamma$ and RV mean SN is almost twice the correlation between the other asymmetry parameters and their corresponding RVs.

Because the median is a more rubust index than the mean, the correlation between $\gamma$ and RV median SN is not as large as the correlation between $\gamma$ and RV mean SN, but it is nonetheless $1.5$ times larger than the correlation between the other common asymmetry parameters and their corresponding RVs. In other words, changes in the asymmetry of the CCF are better captured when using the RV mean SN. The correlation between FWHM and the RVs, either by using RV mean SN or RV median SN, is as well stronger when fitting a SN distribution rather than a Normal. All the correlations are statistically different from $0$. Recalling the analyses presented in Section \ref{sec:soap}, we could infer that Alpha Centauri b is dominated by faculae, because the correlations between the RVs and the width of the CCF are strong (in particular the correlation between RV mean SN and SN FWHM is $0.817$).
%
\begin{figure*}[htbp]
   \centering
\includegraphics[height = 4in]{HD12862_[4]Comparison_para.pdf}  
   \caption{Correlation between the asymmetry parameters and the RVs for Alpha Centauri B. The last three plots show the correlation between the FWHM and the RVs for Alpha Centauri B using respectively the SN (RV mean SN and RV median SN) and the Normal fits. The correlations are always stronger when using parameters derived from the SN fit than the Normal one.}
   \label{fig:alphacent:corrPlot}
\end{figure*}
%

Both the proposed indicators coming from the SN distribution have desirable and undesirable behaviors: RV mean SN better catches changes in the asymmetry of the CCF but the resulting set of RVs ends up being contaminated by those spurious shifts caused by stellar activity that have been shortly presented in Section \ref{intro}. When using RV median SN, the final set of radial velocity is less affected by those spurious shifts caused by stellar activity, but at the same time this indicator is not able to catch as well as RV mean SN changes in the shape and in the width of the CCF. Anyway, both RV mean SN and RV median SN are useful to catch different aspects of the CCF and our suggestion is to use RV mean SN when interested in retrieving information about changes in the shape and/or the width of the CCF. In order to provide a set of RVs containing the smallest amount of spurious contamination imputable to stellar activity, our suggestion is to use instead RV median SN. 

%-----------------------------------------------------------------------------------------------------------------------------------------------
\subsection{Doppler shift added to Alpha Centauri B} \label{sec:soap_real}

We also consider a real-data example using HARPS spectra from the star Alpha Centauri B with an imputed Doppler shift added...

\umberto{to be done}

In the next Section we further motivate the reasons to define the RVs derived by the CCF by calculating RV median SN. In order to do that, we retrieve the standard errors associated with RV mean SN, RV median SN and RV.

%%%%%%%%%%%%%%%%%%%%%%%%%%%%%%%%%%%%%%%%%%%%%%%%%%
\section{Estimation of standard errors for the CCF parameters} \label{sec:5}

In this section, we perform a bootstrap analysis to measure the standard errors associated to RV mean SN, RV median SN, RV, FWHM, SN FWHM, BIS SPAN and $\gamma$. Because a CCF is obtained from a cross-correlation, each point in a CCF is correlated with each other. Therefore, we cannot do a bootstrap analysis on perturbing independently each CCF point with a Gaussian distribution scaled to the error of each given point. To bypass this problem, we bootstrap a hundred times the stellar spectrum given the photon-noise error of each wavelength and calculate for each realization a new CCF. We then fit a Normal or a SN to each of these CCFs and then calculate the standard deviations of the distribution for the location parameters (RV, RV mean SN or RV median SN), the width parameters (FWHM or SN FWHM) and the parameters of asymmetry (BIS SPAN or $\gamma$).
\comment{It is not clear what you are doing here. I can understand bootstrapping a time series: you shuffle all the values keeping the time stamps fixed. How do you bootstrap a stellar spectrum? Do you shuffle the intensity values keeping the wavelengths fixed? That would produce a mess. Do you just add different levels of random noise? But the spectra already have noise in them, they are real observations. You need to state more clearly here what you did.}

\subsection{Estimation of standard errors for the CCF parameters for the simulation study} \label{sec:bootstrap_soap}

We start by calculating the standard errors of the parameters retrieved in Section \ref{sec:soap}, where with SOAP we produced CCFs contaminated by a faculae or a spot. In the third and final case we considered, beyond the spot, a planetary signal that produces pure doppler shifts i the CCFs.

Fig.~\ref{fig:se.soap.faculae} shows the results of the bootstrap analysis performed when a faculae is present on the photosphere of the star. The series of three plots in the top of Fig.~\ref{fig:se.soap.faculae} show the different errors for the RVs, defined as RV (red triangles), RV mean SN (black circles) or RV median SN (cyan crosses), the width and the asymmetry of the CCFs. In the three plots in the bottomof Fig.~\ref{fig:se.soap.faculae} we show the ratio between the parameters derived from the bootstrap analysis fitting the SN and the parameters derived from the bootstrap analysis fitting the Normal distribution. Concerning the standard errors related to the RVs, the ratio between the RV error measured by the bootstrap using the SN and Normal fitting is $1.5$ when using RV mean SN and $0.9$ when using RV median SN. By using RV median SN we get standard errors $10\%$ smaller than using the Normal fit and its corresponding mean. Regarding the errors in width of the CCF, we see that the bootstrap analysis for the Normal and the SN are comparable. Therefore, the precision in the width of the CCF is the comparable if we fit a Normal or a SN to the CCF. Finally, for the errors in evaluating the asymmetry of the CCF, we see that, when fitting the SN to the CCF, the asymmetry errors are $20\%$ smaller. Therefore, the SN fit gives a better precision in CCF asymmetry than what can be reached using BIS SPAN.

\begin{figure*}[htbp]
   \centering
\includegraphics[height = 4in]{RV_comparison_FACULAE_standard_errors.pdf} 
   \caption{Faculae Case. Comparison between the standard errors using the bootstrap analysis for the RVs, the FWHM and the asymmetry parameter. When using RV mean SN (black circles), the standard errors are in average $50\%$ larger than the standard errors retrieved fitting a Normal (red triangles). However, if using RV median SN (cyan crosses), the standard errors are in average $10\%$ smaller than the standard errors coming from the Normal fit. To use as asymmetry parameter $\gamma$ of the SN leads to standard errors in average $20\%$ smaller than the standard errors related to the BIS SPAN. \umberto{explain what happens for those CCF 15 to 19 where s.e. decrease.} Note that for the asymmetry, the error in BIS SPAN is in \kms. To be able to compare the errors in $\gamma$ and BIS SPAN, we multiplied the error in $\gamma$ by the slope of the correlation between $\gamma$ and BIS SPAN.}
   \label{fig:se.soap.faculae}
\end{figure*}

Fig.~\ref{fig:se.soap.spot} shows the results of the bootstrap analysis performed when a spot is present on the photosphere of the star. The series of plots follows the specifications outlined for the previous case. Concerning the standard errors related to the RVs, the ratio between the RV error measured by the bootstrap using the SN and Normal fitting is $1.4$ when using RV mean SN and $0.9$ when using RV median SN. Regarding the errors in width of the CCF, we see that the bootstrap analysis for the Normal and the SN are comparable. Therefore, the precision in the width of the CCF is the comparable if we fit a Normal or a SN to the CCF. Finally, for the errors in evaluating the asymmetry of the CCF, we see that, when fitting the SN to the CCF, the asymmetry errors are $20\%$ smaller.

\begin{figure*}[htbp]
   \centering
\includegraphics[height = 4in]{RV_comparison_SPOT_standard_errors.pdf} 
   \caption{Spot case. Comparison between the standard errors using the bootstrap analysis for the RVs, the FWHM and the asymmetry parameter. When using RV mean SN (black circles), the standard errors are in average $40\%$ larger than the standard errors retrieved fitting a Normal (red triangles). However, if using RV median SN (cyan crosses), the standard errors are in average $10\%$ smaller than the standard errors coming from the Normal fit. To use as asymmetry parameter $\gamma$ of the SN leads to standard errors in average $20\%$ smaller than the standard errors related to the BIS SPAN. Note that for the asymmetry, the error in BIS SPAN is in \kms. To be able to compare the errors in $\gamma$ and BIS SPAN, we multiplied the error in $\gamma$ by the slope of the correlation between $\gamma$ and BIS SPAN.}
   \label{fig:se.soap.spot}
\end{figure*}

Fig.~\ref{fig:se.soap.spot.planet} shows the results of the bootstrap analysis performed when a spot is present on the photosphere of the star. The series of plots follows the specifications outlined for the previous two cases. The conclusions are comparable to the case in which only a spot is present on the photosphere of the star. The ratio between the RV error measured by the bootstrap using the SN and Normal fitting is $1.4$ when using RV mean SN and $0.9$ when using RV median SN. The errors in width of the CCF are comparable and the errors in evaluating the asymmetry of the CCF are $20\%$ smaller when using the asymmetry parameter $\gamma$ of the SN.

\begin{figure*}[htbp]
   \centering
\includegraphics[height = 4in]{RV_comparison_SPOT_PLANET_standard_errors.pdf} 
   \caption{Spot and Planet case. Comparison between the standard errors using the bootstrap analysis for the RVs, the FWHM and the asymmetry parameter. When using RV mean SN (black circles), the standard errors are in average $50\%$ larger than the standard errors retrieved fitting a Normal (red triangles). However, if using RV median SN (cyan crosses), the standard errors are in average $10\%$ smaller than the standard errors coming from the Normal fit. To use as asymmetry parameter $\gamma$ of the SN leads to standard errors in average $20\%$ smaller than the standard errors related to the BIS SPAN. Note that for the asymmetry, the error in BIS SPAN is in \kms. To be able to compare the errors in $\gamma$ and BIS SPAN, we multiplied the error in $\gamma$ by the slope of the correlation between $\gamma$ and BIS SPAN.}
   \label{fig:se.soap.spot.planet}
\end{figure*}

\subsection{Estimation of standard errors for the CCF parameters for real stars} \label{sec:bootstrap_real_star}
In the top plots of Fig.~\ref{fig:se} we show the different errors for the RVs, either defined as RV (red triangles), RV mean SN (black circles) or RV median SN (cyan crosses), the width and the asymmetry of the CCFs for three star, HD215152, HD192310 and Corot-7, that are all at different SNR levels. The parameter SN50 corresponds to the SNR in order 50, which defines a wavelength of 550\,nm. In the bottom plots, we show the ratio between the parameters derived from the bootstrap analysis fitting the SN and the parameters derived from the bootstrap analysis fitting the Normal distribution. We first see that the errors on the CCF parameters only depends on the SNR and do not depend on the spectral type. This is true if the spectral type are not too different though, like here where we show the results for G and K dwarfs.

Concerning the standard errors related to the RVs, the ratio between the RV error measured by the bootstrap using the SN and Normal fitting is $1.6$ when using RV mean SN and $0.9$ when using RV median SN. In other words, by using RV median SN as parameter that defines the radial velocity of the star given a CCF, we get standard errors $10\%$ smaller than using the Normal fit and its corresponding mean. This result is consistent with what we observed with the simulation from SOAP presented in Section \ref{sec:soap}.

Regarding the errors in width of the CCF, we see that the bootstrap analysis for the Normal and the SN are comparable. Therefore, the precision in the width of the CCF is the comparable if we fit a Normal or a SN to the CCF.

Finally, for the errors in evaluating the asymmetry of the CCF, we see that, when fitting the SN to the CCF, the asymmetry errors are $15\%$ smaller. Therefore, the SN fit gives a better precision in CCF asymmetry than what can be reached using BIS SPAN. We recall moreover that, using the SN, all parameters are automatically retrieved in 1 single step, while in the common approach the RV and the BIS SPAN are calculated separately.

%
\begin{figure*}[htbp]
   \centering
\includegraphics[height = 4in]{[5]Errors_vs_SNR_all_stars.pdf} 
   \caption{Comparison between the standard errors using the bootstrap analysis for the RVs, the FWHM and the asymmetry parameter. When using RV mean SN (black circles), the standard errors are in average $60\%$ larger than the standard errors retrieved fitting a Normal (red triangles). However, if using RV median SN (cyan crosses), the standard errors are in average $10\%$ smaller than the standard errors coming from the Normal fit. To use as asymmetry parameter $\gamma$ of the SN leads to standard errors in average $15\%$ smaller than the standard errors related to the BIS SPAN. Note that for the asymmetry, the error in BIS SPAN is in \kms. To be able to compare the errors in $\gamma$ and BIS SPAN, we multiplied the error in $\gamma$ by the slope of the correlation between $\gamma$ and BIS SPAN.}
   \label{fig:se}
\end{figure*}

%, using the spectrum derivative
%, the pure photon-noise error on RV measured 
%Together with the pointwise estimates, retrieving the standard errors for the RVs of the star, $\gamma$ and the FWHM provides information about the uncertainties associated to each parameter. Since the procedure for retrieving the RVs of the star consists in fitting with a density function a CCF, getting the standard errors for all the parameter of interests cannot be done by using the common asymptotic theory. The implementation of a standard bootstrap analysis cannot as well be done, because the points of the CCF are highly correlated. In the analysis of the present work, in order to get the standard errors for the parameters of interest for a particular CCF, we moved each point of this CCF according to its measurement errors, producing $100$ simulated CCFs from the original one. Then we run both the Normal and the SN analyses on these simulated CCFs, letting us to retrieve an evaluation of the standard errors. This operation has been repeated for all the CCFs available for the analyzed stars. The comparison between the standard errors retrieved in this way with the classic noise parameter, {\color{red} obtained by using the Gray's equation cit. Gray (1983)} are presented in Fig.~ \ref{fig:se}.  {\color{red} Xavier: Is it correct how the noise parameter is obtained? Or it is simply $\sqrt(photons)$?}

%Given the strong correlation between the $\gamma$ parameter of the SN and its mean, the standard errors are larger respect the Normal fitting analysis. This fact confirms how correcting the RVs for stellar activity using Eq. \ref{eq:RV:correction} is more helpful when using the SN fitting. The standard errors for the FWHM are comparable for both the SN and the Normal analyses, whereas an improvement is clearly measurable for the standard errors associated to the asymmetry indicators. The last plot in Fig.~\ref{fig:se} shows that the standard errors for $\gamma$ are $10 \%$ smaller than the ones retrieved for the BIS SPAN.

%-----------------------------------------------------------------------------------------------------------------------------------------------
\section{Discussion} \label{sec:discu}

An analysis of the CCF residuals after fitting a Normal or SN distribution shows that the SN is a slightly better model to explain the shape of the CCF. This comes from the fact that CCFs present a natural asymmetry due the convective blueshift.

We tested at first our assumptions by using simulated CCFs retrieved using the software SOAP 2.0. We then compared for five real stars the difference between the RVs (defined as mean of a Normal, mean of the SN or median of the SN), FWHM and asymmetry (BIS SPAN in the Normal case and $\gamma$ in the SN case). The $\gamma$ parameter is linearly dependent on the BIS SPAN, with always a strong correlation coefficient. 
%To compare $\gamma$, that do not have any units, with BIS SPAN expressed in \kms, one has simply to multiply $\gamma$ by the slope of the linear dependence between $\gamma$ and BIS SPAN. 
The slope of this linear correlation changes depending on the studied star. This is probably because the spectral type is different, therefore the effects from stellar activity are different.

When using as parameter for the RV the mean of the SN, the standard errors are in average $60\%$ larger than the standard errors retrieved fitting a Normal. However, once the RV is defined as the median of the SN (cyan crosses), the standard errors are in average $10\%$ smaller than the standard errors coming from the Normal fit. When looking at the correlation between the asymmetry and width parameters of the CCF (FWHM and BIS SPAN or the alternative indicators in \citet{Figueira-2013} in the Normal case, and SN FWHM and $\gamma$ in the SN case) with respect to the RVs (RVs in the Normal case or SN RVs in the SN case), we observe that the correlations are always stronger for the parameters of the SN. Therefore, the SN parameters are more sensitive to activity. In the case of Tau Ceti, which is at very low activity level, we find a significant correlation of $0.322$ between $\gamma$ and RV mean SN, while for all the other asymmetric parameterization, BIS SPAN or the alternative indicators in \citet{Figueira-2013}, the correlations are weaker with a maximum of 0.$225$.

%%%%%%%%%%%%%%%%%%%%%%%%%%%%%%%%%%%%%%%%%%%%%%%%%
\section{Conclusion} \label{sec:conclu}

In this paper we introduced a novel approach based on the Skew Normal (SN) distribution for deriving RVs and shape variations in the CCF of stars. When searching for small-mass exoplanets using the RV technique, it is essential to understand the shape variation of the CCF, which is a proxy for stellar activity effects. The standard approach consist at first to adjust a Normal distribution to the CCF to get the RV and FWHM, defined as the mean and the FWHM of the Normal distribution, and then to measure the asymmetry by calculating BIS SPAN. FWHM and BIS SPAN give us information on the line shape that are used to probe stellar activity signals. 

In this paper we propose to conduct the analysis fitting a SN distribution to the CCF. Since the CCF presents a natural asymmetry due the convective blueshift, the SN distribution can better catch these aspects respect the Normal fit. On top of that, by using the SN distribution to fit CCFs, we can measure simultaneously the RV of the star, the width and the asymmetry of the CCF.

Starting from the simulation environment SOAP and then moving to real stars, we showed that using the SN distribution to fit CCFs brings a significant improvement in probing stellar activity. While for the Normal distribution mean and median are equivalent, using the SN fit different location parameters can be tested. While the median of the SN is more robust respect variations in the shape of the CCF, the mean of the SN is more sensible to changes in the asymmetry of the CCF. We suggest to use as parameter that defines the RV of the star the median of the SN, since the standard errors related to this parameter are $10\%$ smaller than the standard errors retrieved using the Normal distribution. For evaluating changes in the asymmetry of the CCF, we suggest to use the mean of the SN. The correlations between RV mean SN and SN FWHM, and RV mean SN and $\gamma$ (the asymmetry parameter of the SN) are much stronger than the correlations between the equivalent parameters derived using a Normal fit (RV, FWHM and BIS SPAN or the asymmetric parameters described in \citet{Figueira-2013}). The precision on the asymmetry measured by $\gamma$ is greater than the one on BIS SPAN by $\sim$15\%. Therefore when searching for rotational periods in the data, or applying Gaussian Processes to account for stellar activity signals, the SN parameters should be used.

Finally, we also encourage the use of bootstrapping to estimate more realistic errors on the different parameters of the Normal or SN fitted to the CCF, mainly in the low SNR regime where a gain of 50\% can be reached. This takes significantly more time, but note that 100 realization are enough to get a good estimation of errors.


%the RVs of the star can be calculated as the mean of the SN, whereas its asymmetry parameter $\gamma$ can be naturally used for retrieving information about the asymmetry of the CCF. Because of the natural correlation between the RVs and the asymmetry parameter, a linear combination of both $\gamma$ and the FWHM has been implemented for correcting the RVs from stellar activity, allowing us to retrieve a more precise RVs. 

%In order to get the standard errors for the estimated parameters, rather than using the common $noise$ statistic, we created a simulated procedure based on perturbing each point of the CCF according to its measurement error. Then, by getting from the original profile line $100$ simulated profiles, we run the SN fitting analyses, allowing us retrieving for all the parameter of interests an estimation for their standard errors. Concerning the standard error associated to the means of the distributions, the SN fit leads to slightly larger standard errors respect when running the Normal fit. This fact can be explained by noticing that as a simulated CCF is created according to the procedure described above, we expect slightly differences on the asymmetry of the CCF as well. The natural correlation between the RVs of the star and $\gamma$, where the mean of the SN is shifted in the direction of the asymmetry of the distribution, leads to larger uncertainties related to the RVs of the star. The latter consideration provides a further justification for correcting the RVs of the star from stellar activity and explains why this correction is more helpful when running the proposed analysis rather than the classic Normal fit. Concerning the standard errors related to the asymmetry parameters, when using the SN fit the uncertainties are $10 \%$ smaller than using the classic BIS SPAN. In conclusion, we showed how using the SN fit provides in an unique step a better understanding of the stellar activity through using the asymmetry parameter $\gamma$ rather than using the common statistics such as the BIS SPAN or other indicators. On top of that, when focusing on low signal-to-noise cases, the SN procedure showed how statistically significant results are retrieved with the proposed approach in terms of correlation between the asymmetry of the profile line and the RVs, whereas for the previous analyses this correlation does not result statistically significant.

%-----------------------------------------------------------------------------------------------------------------------------------------------
\section{Acknowledgements}

We are grateful to all technical and scientific collaborators of the HARPS Consortium, ESO Headquarters and ESO La Silla who have contributed with their extraordinary passion and valuable work to the success of the HARPS project.
XD is grateful to the Society in Science--The Branco Weiss Fellowship for its financial support.


%-----------------------------------------------------------------------------------------------------------------------------------------------
%\section{Appendix}
\appendix
\section{Appendix} \label{appendix}

In this Appendix we present the analyses conducted on other 4 stars: HD192310, HD10700, HD215152 and finally Corot-7. 

\umberto{Add further information about the stars here presented.}

Table \ref{table:summaryStars} summarizes the results obtained by the SN fit and the some of the results based on the Normal fit. The results are all consistent with the conclusions derived by the analyses on Alpha Centauri b. The correlation between $\gamma$ and RV mean SN is stronger than the correlation between the BIS SPAN and RV for all the considered stars. The correlation between SN FWHM and RV mean SN is stronger than the correlation between FWHM and RV for three of the four stars. The analyses of the correlations for the four stars are are presented Fig. \ref{fig:Gliese785:corrPlot}, \ref{fig:Tau:corrPlot}, \ref{fig:HD215152:corrPlot} and \ref{fig:Corot7:corrPlot}.

\small{
\begin{table}[!t]
\scalebox{0.45}{
\begin{tabular}{|c|c|c|c|c|c|c|c|c|}
\hline
\textbf{Star}          &\textbf{ \# CCFs}  &   \textbf{$\text{R}(\text{SN }\gamma, \text{Bis-Span})$} & \textbf{$\text{slope}(\text{SN }\gamma, \text{Bis-Span})$} &   \textbf{$\text{R}(\text{SN }\gamma, \text{RV mean SN})$} & \textbf{$\text{R}(\text{Bis-Span}, \text{RV})$} & \textbf{$\text{R}(\text{FIG BiGaussian}, \text{RV})$} & \textbf{$\text{R}(\text{SN FWHM}, \text{RV mean SN})$}  & \textbf{$\text{R}(\text{FWHM}, \text{RV})$} \\
\hline
 $\text{HD}192310  $          &    $1577$    & $0.888$ & $0.786$ & $0.669 (0.64; 0.695)$ & $0.329 (0.285; 0.373)$  & $-0.333 (-0.376; -0.289)$ & $0.666 (0.637; 0.692)$ & $0.476 (0.4367 0.514)$\\
\hline
 $\text{HD}10700 $            &    $7928$    & $0.78$ & $0.604$ & $0.322 (0.302; 0.342)$ & $-0.073 (-0.095; -0.0051)$ & $0.127 (0.105; 0.148)$ & $0.421 (0.403; 0.439)$ & $0.529 (0.513; 0.545)$ \\
\hline
 $\text{HD}215152 $          &     $273$   &  $ 0.763$ & $0.794$ & $0.571 (0.485; 0.646)$ & $-0.067 (-0.184; 0.052)$  & $0.269 (0.155; 0.376)$ & $0.210 (0.094; 0.321)$ & $-0.138 (-0.253; -0.020)$ \\
\hline
 $\text{Corot }7$     &     $173$    &  $0.814$  & $0.607$ & $0.561 (0.450; 0.656)$ & $0.092 (-0.058; 0.238)$ & $-0.335 (-0.228; -0.082)$ & $-0.709 (0.626;0.776)$ & $0.595 (0.489; 0.683)$ \\
\hline
\end{tabular}
}
\caption{Subset of notable correlations between the asymmetry parameter (and the FWHM) and the RVs for four stars: $\text{HD}192310$,  $\text{HD}10700$, $\text{HD}215152$ and $\text{Corot }7$.}
\label{table:summaryStars}
\end{table}
}

\begin{figure*}[htbp]
   \centering
\includegraphics[height = 4in]{HD19231_[4]Comparison_para.pdf} 
   \caption{Correlation between the asymmetry parameters and the RVs for HD 192310. The last three plots show the correlation between the FWHM and the RVs for HD 192310 using respectively the SN and the Normal fits. \umberto{faculae???}}
   \label{fig:Gliese785:corrPlot}
\end{figure*}

\begin{figure*}[htbp]
   \centering
\includegraphics[height = 4in]{HD10700_[4]Comparison_para.pdf}  
   \caption{Correlation between the asymmetry parameters and the RVs for Tau Ceti. The last three plots show the correlation between the FWHM and the RVs for Tau Ceti using respectively the SN and the Normal fits.\umberto{faculae???}}
   \label{fig:Tau:corrPlot}
\end{figure*}

\begin{figure*}[htbp]
   \centering
\includegraphics[height = 4in]{HD21515_[4]Comparison_para.pdf}  
   \caption{Correlation between the asymmetry parameters and the RVs for HD 215152. The last three plots show the correlation between the FWHM and the RVs for HD 215152 using respectively the SN and the Normal fits.\umberto{spot???}}
   \label{fig:HD215152:corrPlot}
\end{figure*}

\begin{figure*}[htbp]
   \centering
\includegraphics[height = 4in]{LRa01_E_[4]Comparison_para.pdf} 
   \caption{Correlation between the asymmetry parameters and the RVs for Corot-7. The last three plots show the correlation between the FWHM and the RVs for Corot-7 using respectively the SN and the Normal fits.\umberto{faculae???}}
   \label{fig:Corot7:corrPlot}
\end{figure*}

%%%%%%
\iffalse
%\subsection{HD192310}  \label{sec:Gl785}
%
\begin{figure*}[htbp]
   \centering
\includegraphics[height = 3in]{[0]HD19231_HistogramsDiff.pdf} 
   \caption{RVs comparison for HD192310 considering a Normal and a SN fitting (using both the mean and the median).}
   \label{fig:HD192310:RV}
\end{figure*}
%
\begin{figure}[htbp]
   \centering
\includegraphics[height = 3in]{HD19231_[2]gamma_vs_bisspan.pdf} 
   \caption{Correlation between $\gamma$ and the BIS SPAN for HD192310.}
   \label{fig:Gliese785:corr.gamma}
\end{figure}
%


%\subsection{Tau Ceti}  \label{sec:Taucet}
%
\begin{figure*}[htbp]
   \centering
\includegraphics[height = 3in]{[0]TauCeti_HistogramsDiff.pdf} 
   \caption{RVs comparison for Tau Ceti considering a Normal and a SN fitting (using both the mean and the median).}
   \label{fig:Tau Ceti:RV}
\end{figure*}
%
\begin{figure}[htbp]
   \centering
\includegraphics[height = 3in]{HD10700_[2]gamma_vs_bisspan.pdf} 
   \caption{Correlation between $\gamma$ and the BIS SPAN for Tau Ceti.}
   \label{fig:Tau:corr.gamma}
\end{figure}
%


%\subsection{HD215152}  \label{sec:HD215152}
%
\begin{figure*}[htbp]
   \centering
\includegraphics[height = 3in]{[0]HD21515_HistogramsDiff.pdf} 
   \caption{RVs comparison for HD215152 considering a Normal and a SN fitting (using both the mean and the median).}
   \label{fig:HD215152:RV}
\end{figure*}
%
\begin{figure}[htbp]
   \centering
\includegraphics[height = 3in]{HD21515_[2]gamma_vs_bisspan.pdf} 
   \caption{Correlation between $\gamma$ and the BIS SPAN for HD 215152.}
   \label{fig:HD215152:corr.gamma}
\end{figure}
%


%\subsection{Corot-7}  \label{sec:Corot7}
%
\begin{figure*}[htbp]
   \centering
\includegraphics[height = 3in]{[0]LRa01_E_HistogramsDiff.pdf} 
   \caption{RVs comparison for Corot-7 considering a Normal and a SN fitting (using both the mean and the median).}
   \label{fig:corot7:RV}
\end{figure*}
%
\begin{figure}[htbp]
   \centering
\includegraphics[height = 3in]{LRa01_E_[2]gamma_vs_bisspan.pdf} 
   \caption{Correlation between $\gamma$ and the BIS SPAN for Corot-7.}
   \label{fig:Corot7:corr.gamma}
\end{figure}
%
\fi
%

%%-----------------------------------------------------------------------------------------------------------------------------------------------
\bibliographystyle{aa}
%\bibliography{dumusque_bibliography}
\bibliography{mybib-SNCCF}

%\begin{appendix}
%\end{appendix}

\end{document}
