\documentclass{aa}
%\documentclass[referee]{aa}
\usepackage{natbib}
%\usepackage{lscape}
\usepackage{pdflscape} % or {lscape}
\usepackage{wasysym}
\usepackage[varg]{txfonts}
\usepackage{graphicx}
\usepackage{hyperref}
\bibpunct{(}{)}{;}{a}{}{,} % to follow the A&A style
\usepackage{longtable}

\usepackage{afterpage}


\usepackage[usenames,dvipsnames]{color}
\definecolor{mygreen}{rgb}{0,0.5,0}
\definecolor{myorange}{rgb}{0.5,0.5,0}
\definecolor{myred}{rgb}{0.5,0,0}

%\newcommand{\ep}[1]{\textcolor{red}{#1}}
%\newcommand{\epp}[1]{\textcolor{blue}{#1}}

\newcommand{\fo}{f_\mathrm{o}}
\newcommand{\fu}{f_\mathrm{u}}
\newcommand{\ic}{I_\mathrm{c}}
\newcommand{\rc}{R_\mathrm{c}}
\newcommand{\uspec}{U_\mathrm{spec}}
\newcommand{\lambdac}{\lambda_\mathrm{c}}
\newcommand{\mplanet}{M_\mathrm{p}}
\newcommand{\mearth}{M_\oplus}
\newcommand{\rearth}{R_\oplus}
\newcommand{\msun}{M_\odot}
\newcommand{\rsun}{R_\odot}
\newcommand{\mstar}{M_\star}
\newcommand{\rstar}{R_\star}
\newcommand{\mjup}{M_\mathrm{jup}}
\newcommand{\rjup}{R_\mathrm{jup}}
\newcommand{\rplanet}{R_\mathrm{p}}
\newcommand{\rplanetzero}{R_\mathrm{p,0}}
\newcommand{\rhoplanet}{\rho_\mathrm{p}}
\newcommand{\de}{\mathrm{d}}
\newcommand{\teq}{T_\mathrm{eq}}
\newcommand{\kb}{k_\mathrm{b}}
\newcommand{\htwo}{\mathrm{H}_2}
\newcommand{\htwoo}{\mathrm{H}_2\mathrm{O}}
\newcommand{\chfour}{\mathrm{CH}_4}
\newcommand{\der}{\de\rplanet/\de\ln\!\lambda }
\newcommand{\chisq}{\chi^2}
\newcommand{\chisqr}{\chi^2_\mathrm{r}}
\newcommand{\teff}{T_\mathrm{eff}}
\newcommand{\logg}{\log g}
\newcommand{\feh}{[\mathrm{Fe}/\mathrm{H}]}

\def\ms{\hbox{\,m\,s$^{-1}$}}         %m.s -1
\def\cms{\hbox{\,cm\,s$^{-1}$}}       %cm.s -1
\def\m2s2{\hbox{\,m$^{2}$\,s$^{-2}$}} %m2.s -2
\def\kms{\hbox{\,km\,s$^{-1}$}}       %km.s -1
\def\vsini{\hbox{$v$\,sin\,$i$\,}}      %vsini
\def\sini{\hbox{sin\,$i$}}      %vsini
\def\Msun{\hbox{$\mathrm{M}_{\odot}$}}             %Msun
\def\Rsun{\hbox{$\mathrm{R}_{\odot}$}}
\def\Mjup{\hbox{$\mathrm{M}_{\rm Jup}$}}
\def\Rjup{\hbox{$\mathrm{R}_{\rm Jup}$}}
\def\degr{\hbox{$^\circ$}}
\def\chisq{\mbox{$\chi^2$}}
%\def\mp{$M_{\rm p}$}
%\def\rp{$R_{\rm p}$}
\def\mp{M_{\rm p}}
\def\rp{R_{\rm p}}
\def\logrhk{$\log$(R$^{\prime}_{HK}$)}


\newcommand{\jessi}[1]{{\color{Purple}[[\textbf{Jessi: }#1]]}}
\newcommand{\xavier}[1]{{\color{blue}[[\textbf{Xavier: }#1]]}}
\newcommand{\umberto}[1]{{\color{green}[[\textbf{Umberto: }#1]]}}
%\newcommand{\comment}[1]{{\color{red}[[\textbf{Referee: }#1]]}}
\newcommand{\todo}[1]{{\color{cyan}[[\textbf{TODO: }#1]]}}


\begin{document}

\title{Measuring precise radial velocities and cross-correlation function line-profile variations using a Skew Normal density
\thanks{Based on observations collected at the La Silla Parana Observatory,
ESO (Chile), with the HARPS spectrograph at the 3.6-m telescope.}}


\author{U. Simola \inst{1,2}
	    \thanks{\email{umberto.simola@helsinki.fi}}
	    \and X. Dumusque\inst{3}
	    \thanks{Branco Weiss Fellow--Society in Science (url: \url{http://www.society-in-science.org})}    
	    \and Jessi Cisewski\inst{2}
	    }

\institute{Department of Statistical Sciences, University of Padua, Padua, Italy
	      \and Department of Statistics and Data Science, Yale University, New Haven, CT, USA
	      \and Observatoire de Gen\`eve, Universit\'e de Gen\`eve, 51 ch. des Maillettes, CH-1290 Versoix, Switzerland 
	      }

\date{Received XXX; accepted XXX}

\abstract
% Context, Aims, Methods, Results, Conclu (not mandatory)
{Stellar activity is one of the primary limitations to the detection of low-mass exoplanets using the radial-velocity (RV) technique. 
Stellar activity can be probed by measuring time dependant variations in the shape of the cross-correlation function (CCF), often estimated using the different moments of the modelled CCF. Therefore estimating the moments of the CCF with high precision is essential to de-correlate exoplanet signals from spurious RV signals originating from stellar activity.}
%
{We propose to estimate the moments of the CCF by fitting a model using a Skew Normal (SN) density shape, which unlike the commonly employed Normal density, \xavier{includes an skewness parameter to capture the asymmetry of the CCF induced by stellar activity, but also the natural asymmetry induced by convective blueshift.}}
%
{The performance of the proposed method is compared to the Normal density using both simulated and real observations with varying levels of activity and signal-to-noise ratio (SNR) levels.}
%We analyze 5 stars with different activity levels and whose CCF's have different signal-to-noise ratio (SNR) levels. In each case, we compare the results obtained by fitting to the CCF respectively a Normal and a SN.  We also estimate rigorous errors for the different moments of the CCF using a bootstrap analysis.
%
{When considering the real observations, the correlation between the RV's and the asymmetry of the CCF and the correlation between the RV's and the width of the CCF are stronger when using the parameters derived from the SN than the Normal approach. 
This suggests that the CCF asymmetry and the CCF width derived using a SN may be more sensitive to stellar activity, which can be helpful with estimating stellar rotational periods and generally characterizing the stellar activity signals.
The estimated uncertainties in the estimated RV's using the proposed SN approach are on average $10\%$ smaller than the uncertainties calculated on the mean of the Normal, and the estimated uncertainties on the SN asymmetry parameter are on average $15\%$ smaller than the commonly used Bisector Inverse Slope Span (BIS SPAN) approach for estimating the asymmetry of the CCF. }
%
%{The derived benefits of using the proposed SN approach includes an asymmetry parameter along with several options for estimating the RV, while the Normal model only includes an RV estimate and requires a separate procedure for estimating the asymmetry.  Furthermore, the estimated uncertainties on the RV's and the asymmetry parameter are smaller in the SN setting.}
%The correlation between the RV's and the asymmetry of the CCF and the correlation between the RV's and the width of the CCF are always stronger when using the parameters derived from the SN in the case of real observations. Therefore the CCF asymmetry and the CCF width derived using a SN are more sensitive to stellar activity, which allows to probe with a better precision stellar rotational periods, in addition to characterize more precisely stellar activity signals. The precision on the estimated set of RV's, derived using the median of the fitted SN densities are on average $10\%$ smaller than the uncertainties calculated on the mean of the Normal. In addition, the uncertainties related to the asymmetry parameter derived from the SN are on average $15\%$ smaller than the ones calculated on the common Bisector Inverse Slope Span (BIS SPAN). We strongly encourage the use of the SN density rather than the Normal density, because this allows us to retrieve in one single fit the different moments of the CCF, because the derived moments better catch stellar activity signals and finally because the standard errors on the RV's and the the asymmetry parameter are smaller than the one estimated with the Normal fit.

\keywords{techniques: radial velocities -- planetary systems -- stars: activity -- methods: data analysis}


\titlerunning{Fitting a SN distribution to CCF}
\authorrunning{U. Simola, X. Dumusque and J. Cisewski}
\maketitle

%-----------------------------------------------------------------------------------------------------------------------------------------------
\section{Introduction} \label{intro}

%%%%%%%%%%%%%%%%%%%%%%%%%%%%%%%%%%
%\subsection{Goal of RV analysis}

When working with radial velocities (RV's), one of the main limitations to the detection of small-mass exoplanets is no longer the precision of the instruments used, but the different sources of variability induced by the stars \citep[e.g.][]{Feng:2017aa, Dumusque:2017aa, Rajpaul-2015, Robertson-2014}. 
Stellar oscillations, granulation phenomena, and stellar activity can all induce apparent RV signals \citep[e.g.][]{Saar-1997b, Queloz-2001, Desort-2007, Dumusque-2011a, Dumusque-2016a} that are above the meter-per-second precision reached by the best high-resolution spectrographs \citep[HARPS, HARPS-N,][]{Mayor-2003,Cosentino-2012}.
%
It is therefore mandatory to better understand stellar signals and to develop methods to correct for them, if in the near future we want to detect or confirm an Earth-twin planet using the RV technique. This is even more true now that instrument like the Echelle SPectrograph for Rocky Exoplanet and Stable Spectroscopic Observations (ESPRESSO) \citep{Pepe-2014} and EXtreme PREcision Spectrometer (EXPRES) \citep{fischer2016state} should have the stability to detect such signals. However, if solutions are not found to mitigate the impact of stellar activity, the detection or confirmation of potential Earth-twins will be extremely challenging and false detections could plague the field.

%%%%%%%%%%%%%%%%%%%%%%%%%%%%%%%%%%
%\subsection{Stellar activity effect on CCF, Normal fit plus FWHM and BIS SPAN indicators}

One of the most challenging stellar signal to characterize and to correct for is the signal induced by stellar activity. 
Stellar activity is responsible for creating magnetic regions on the surface of stars, and those regions change locally the temperature and the convection, which can induce spurious RV's variations \citep{Meunier-2010a, Dumusque-2014b}. 
In theory, it should be easy to differentiate between the pure Doppler-shift induced by a planet, which shifts the entire stellar spectrum, and stellar activity, which modifies the shape of spectral lines and by doing so create a spurious shift of the stellar spectrum \citep{Saar-1997b,Hatzes-2002,Kurster2003,Lindegren-2003,Desort-2007,Lagrange-2010,Meunier-2010a,Dumusque-2014b}. 
However, on quiet GKM dwarfs, the main target for precise RV's measurements, stellar activity can induce signals of a few \ms. This corresponds physically to variations smaller than 1/100th of a pixel on the detector making the changing shape of the spectral lines challenging to detect.

In order to measure such tiny variations, a common approach is to average the information of all the lines in the spectrum by cross correlating the stellar spectrum with a synthetic \citep[][]{Baranne-1996,Pepe-2002a} or an observed stellar template \citep[][]{Anglada-Escude-2012}. The result of this operation gives us the cross-correlation function (CCF).  The CCF gives the spectrum's cross-correlation with the template as the template is shifted according to different RVs.
%
To measure the Doppler-shift between different spectra and therefore to retrieve the RV's of a star as a function of time, the variations of the CCF barycenter are calculated. 
The barycenter is generally estimated by fitting a Normal density to the CCF and retaining its mean. Variations in line shape between different spectra, which indicate the presence of signals induced by stellar activity, are measured by analyzing the different moments of the CCF. Usually, the width of the CCF is estimated using the full-width half-maximum (FWHM) of the fitted Normal density, and its asymmetry using the the bisector inverse slope span \citep[BIS SPAN,][]{Queloz-2001}.

%%%%%%%%%%%%%%%%%%%%%%%%%%%%%%%%%%
%\subsection{Why FWHM and BIS SPAN important?}

If an apparent RV signal is induced by activity, generally a strong correlation will be observed between the RV and chromospheric activity indicators like \logrhk\,or H-$\alpha$ \citep{Boisse-2009,Dumusque-2012,Robertson-2014}, but also between the RV and the FWHM of the CCF or its BIS SPAN \citep[][]{Queloz-2001,Boisse-2009,Queloz-2009,Dumusque-2016a}. 
%
It is therefore common now, that when fitting a Keplerian signal to a set of RVs to look for a planet, the model includes in addition linear dependancies with the \logrhk, the FWHM and the BIS SPAN \citep{Dumusque:2017aa,Feng:2017aa}.
%
It is also common to add a Gaussian process to the model to account for the correlated noise induced by stellar activity. The hyperparameters of the Gaussian process can be trained on different activity indicators \citep{Haywood-2014,Rajpaul-2015}. It is therefore essential for mitigating stellar activity to obtain activity indicators that are the most correlated with the RV's but also for which we can obtain the best precision.

%%%%%%%%%%%%%%%%%%%%%%%%%%%%%%%%%%
%\subsection{Figueira indicators of stellar activity + other}
Several indicators have been developed that are more sensitive to line asymmetry than the BIS SPAN. In \citet{Boisse-2011}, the authors develop $V_{span}$, which is the difference between the RV measured respectively by fitting a Normal density to the upper and the bottom part of the CCF. This CCF asymmetry parameter is shown to be more sensitive than the BIS SPAN at low signal-to-noise ratio (SNR).
%
\citet{Figueira-2013} studied the use of two new indicators, bi-Gauss and $V_{asy}$. The authors were able to show that when using bi-Gauss, the amplitude in asymmetry is 30\% larger than when using BIS SPAN, therefore allowing the detection of lower levels of activity. They also demonstrated that $V_{asy}$ seems to be a better indicator of line asymmetry at high SNR, as its correlation with RV is more significant than any of the previously proposed asymmetry indictors.

%%%%%%%%%%%%%%%%%%%%%%%%%%%%%%%%%%
%\subsection{Why using a SN density?}
In all the methods described above, except bi-Gauss, the RV and the FWHM are derived using a Normal density fitted to the CCF, and the asymmetry is estimated using another approach. 
%
In this paper we propose to use a Skew Normal (SN) density to estimate with a single fit of the CCF, the RV, the FWHM and the asymmetry of the CCF, as this function includes a skewness parameter \citep[][]{Azzalini1985}. 

%In addition, we know that for solar-type stars and cooler dwarfs, the bisector of the CCF has a "C"-shape due to convective blueshift \citep{Dravins-1981, Gray-2009}. 
%Therefore, fitting the CCF using a model that naturally includes an asymmetry, like the SN density, should give in principle more precise results.

%%%%%%%%%%%%%%%%%%%%%%%%%%%%%%%%%%
%\subsection{Outline of paper}
The paper is organized as follow. In Sec.~\ref{sec:2} we introduce the SN density, describe its applicability for modeling the CCF, and study how the SN parameters relate to the RV, FWHM and BIS SPAN of the CCF. 
%
In Sec.~\ref{sec:31} we propose an expanded linear model used to correct the estimated RV's from the inferred stellar activity, which extends the linear models previously proposed for this purpose \citep{Dumusque:2017aa,Feng:2017aa}. 
%
In Sec.~\ref{sec:soap} the performance of the SN fit to the CCF is investigated using simulations coming from the Spot Oscillation And Planet (SOAP) 2.0 \citep{Dumusque-2014b}, followed by an analysis of real observations in Sec.~\ref{sec:4}.
%
Sec.~\ref{sec:5} considers derived error bars for the different estimated CCF parameters, and finally a discussion of the results and conclusions are included in Sec.~\ref{sec:discu} and Sec.~\ref{sec:conclu}, respectively.

%-----------------------------------------------------------------------------------------------------------------------------------------------
\section{The Skew Normal distribution} \label{sec:2}

The Skew Normal (SN) distribution is a class of probability distributions which includes the Normal distribution as a special case \citep{Azzalini1985}. The SN distribution has, in addition to a location and a scale parameter analogous to the Normal distribution's mean and standard deviation, a third parameter which describes the skewness (i.e. the asymmetry) of the distribution. Considering a random variable $Y\in \mathbb R$ (where $\mathbb R$ is the real line) which follows a SN distribution with location parameter $\xi \in \mathbb R$, scale parameter $\omega \in \mathbb R^{+}$ (i.e., the positive real line), and skewness parameter $\alpha \in \mathbb R$, its density at some value $y\in Y$ can be written as 
\begin{equation} \label{def:snd_gen}
SN(y;\xi, \omega, \alpha) = \frac{2}{\omega} \phi\left(\frac{y-\xi}{\omega}\right) \Phi\left(\frac{\alpha(y-\xi)}{\omega}\right),
\end{equation}
where $\phi$ and $\Phi$ are respectively the density function and the distribution function of a standard Normal distribution\footnote{A standard Normal distribution is a Normal distribution with a mean of 0 and a standard deviation of 1.}.
The skewness parameter $\alpha$ quantifies the asymmetry of the SN. 
Examples of SN densities under different skewness parameter values and the same location and scale parameters ($\xi = 0$ and $\omega = 1$) are displayed in Fig.~\ref{fig:SN.plot}.  A usual Normal distribution is the special case of the SN distribution when the skewness parameter $\alpha$ is equal to zero\footnote{This can be seen from Eq.~\ref{def:snd_gen}. If $\alpha = 0$ then $\Phi\left(\frac{\alpha(y-\xi)}{\omega}\right) = \Phi(0) = 0.5$ and therefore $SN(y;\xi, \omega, 0) = \frac{1}{\omega} \phi\left(\frac{y-\xi}{\omega}\right)$ which is the density of a Normal distribution. Note that $\Phi(0) = 0.5$ because $\Phi(0)$ is the the probability that a standard Normal random variable is less than or equal than 0.}.
%
\begin{figure}[htbp]
   \centering
\includegraphics[height = 2.3in]{Skew_Normal_densities_jjck.pdf} 
   \caption{Density function of a random variable Y following the SN distribution $SN(\xi, \omega^{2}, \alpha)$ with location parameter $\xi = 0$, scale parameter $\omega = 1$ and different values of the skewness parameter $\alpha$ indicated by different colors and line types. Note that the solid black line has an $\alpha = 0$, making it a Normal distribution.}
   \label{fig:SN.plot}
\end{figure}
%
For reasons related to the interpretation of the parameters in Eq.~\ref{def:snd_gen} and computational issues with estimating $\alpha$ near 0, a different parametrization is used in this work, which is referred to as the \emph{centered parametrization} (CP).  This CP is much closer to the parametrization of a Normal distribution, as it uses a mean parameter $\mu$, a variance parameter $\sigma^2$ and a skewness parameter $\gamma$. In order to define the CP, we need to express the CP parameters $(\mu, \sigma^2, \gamma)$ as a function of $(\xi, \omega^2, \alpha)$. This can be done using the following relations:
%
\begin{equation} \label{eq:snd_cp}
\mu = \xi + \omega \beta, \quad \sigma^{2} = \omega^{2}(1-\beta^2), \quad \gamma = \frac{1}{2}(4-\pi) \beta^{3}\left(1-\beta^2\right)^{-3/2},
\end{equation}
%
where $\beta = \sqrt{\frac{2}{\pi}}\left(\frac{\alpha}{\sqrt{1+\alpha^2}}\right)$ \citep[e.g.][]{Arellano-2010}.

By using Eq.~\ref{eq:snd_cp}, the new set of parameters $(\mu, \sigma^2, \gamma)$ provides a clearer interpretation of the behavior of the SN distribution. For the $\alpha$ values used in Fig.~\ref{fig:SN.plot}, the corresponding values of ($\mu$, $\sigma^2$, $\gamma$) are displayed in Table~\ref{tab:cp_values}.  In particular, $\mu$ and $\sigma^2$ are the actual mean and variance of the distribution, rather than simply a location and scale parameter, and $\gamma$ provides an measure of the skewness of the SN. 
Along with the mean of the SN, we consider the median of the distribution as a measure of center, which is used in the proposed method.  See Table~\ref{tab:cp_values} for the medians of the SN densities displayed in Fig.~\ref{fig:SN.plot}.
%The median of the SN, and in general the median of an absolute continuous random variable, is defined as the value $m$ such that\footnote{We recall that when using a symmetric distribution such as the Normal distribution, the mean and the median are equivalent.}:
%%
%\begin{equation} \label{eq:snmed}
%\int_{-\infty}^{m} SN(y;\xi, \omega, \alpha) = \frac{1}{2}.
%\end{equation}
%

%% Requires the booktabs if the memoir class is not being used
\begin{table}[htbp]
   \centering
   %\topcaption{Table captions are better up top} % requires the topcapt package
   \begin{tabular}{|ccccc|} % Column formatting, @{} suppresses leading/trailing space
\hline
$\alpha$ & $\mu$ & $\sigma^2$ & $\gamma$  & Median \\
\hline
 -3 	&	 -0.757	&	 0.427	&	 -0.667  	& 	-0.672\\
0	&	 0.000 	&	1.000	&	 0.000 	& 	0.000\\
2	&	 0.714	&	 0.491	&	 0.454 	& 	0.655\\
6	&	 0.787	&	 0.381	&	 0.891 	& 	0.674\\
10	&	 0.794	&	 0.370	&	 0.956 	& 	0.674\\
\hline
   \end{tabular}
   \caption{CP values $(\mu, \sigma^2, \gamma)$ along with the median corresponding to the $\alpha$ values shown in Fig.~\ref{fig:SN.plot}, with location parameter $\xi = 0$ and scale parameter $\omega = 1$. Values are rounded to three decimal places.}
   \label{tab:cp_values}
\end{table}
%
Further details about the parametrization from Eq.~\ref{def:snd_gen}, called the \emph{Direct Parametrization} or DP, the CP, and general statistical properties of the SN are treated in rigorous mathematical and statistical viewpoints in the book by \cite{Azzalini2014}.

%-----------------------------------------------------------------------------------------------------------------------------------------------
\subsection{Fitting the Skew Normal density to the CCF} \label{sec:3}

%The CCF represents the average shape of spectral lines and is expressed in flux as a function of radial-velocity.
To fit the CCF using a SN density shape, we use a least-squares algorithm and the following model:
%
\begin{eqnarray} \label{eq:3}
f_{CCF}(x_i) = \mathrm{C} - \mathrm{A} \times SN(x_i;\mu, \sigma^2, \gamma), \quad i = 1, \ldots, n
\end{eqnarray}
%
where C is an unknown offset for the continuum of the CCF, A is the unknown amplitude of the CCF, commonly referred to as the contrast, and $\mu$, $\sigma^2$ and $\gamma$ are the mean, variance and skewness of the SN as defined above.
The values $x_1, \ldots, x_n$ are the different values of the x-axis of the CCF, generally in the unit of a velocity.

%%%Since the CCF has an asymmetry due the convective blueshift, the SN density should in principle better catch this aspect, together with other changes in asymmetry, with respect to fitting a Normal density. 
%%%To initially check this intuition, we compared the CCF residuals after fitting a Normal and a SN density for 2 stars. The first star is Alpha Centauri b, whose CCF's have high signal-to-noise ratio (SNR). The second star is Corot-7, whose CCF's have low SNR. Fig.~\ref{fig:Residual.comparison} shows that the SN seems to be a slightly better model to explain the shape of the CCF.%, in particular as the SNR decreases.
%%%%
%%%\begin{figure*}[htbp]
%%%   \centering
%%%\includegraphics[height = 2.5in]{[1]HD128621Residuals.pdf} 
%%%\includegraphics[height = 2.5in]{[1]LRa01_E2Residuals.pdf} 
%%%   \caption{Comparison between the Normal (black circles) and the SN (red crosses) residuals using CCF's from the star Alpha Centauri b (left) and Corot-7 (right). When looking at the residuals corresponding to the center of the CCF, the SN fit leads to slightly better results for both stars.} %Moreover, as the SNR decreases, the SN density shows smaller residuals respect the Normal ones.}
%%%    \label{fig:Residual.comparison}
%%%\end{figure*}
%%%%

When fitting a Normal density to the CCF, the estimated mean of the model is used as the estimated RV, the FWHM of the Normal density\footnote{FWHM$=2\sqrt{2\ln2}\,\sigma$ with standard deviation $\sigma$} represents the width of the CCF.
Because the Normal density is symmetric, the skewness is always equal to 0 so a separate approach is needed to estimate the skewness of the CCF.
An estimated skewness parameter is generally obtained by calculating the BIS SPAN of the CCF \citep[see Sect. \ref{intro}, and e.g.][]{Queloz-2001}. 
%

With the proposed SN approach, we propose two estimators of the RV: the mean and median of the SN model fit (referred to as SN mean RV and SN median RV, respectively), and present advantages and limitations for both of these choices in Sec. \ref{sec:4} and Sec. \ref{sec:5}. 
The width of the SN, SN FWHM, is defined in the same way as for the Normal density\footnote{Note that SN FWHM does not correspond to the width of the SN density at half maximum like in the Normal case.}, and finally the skewness of the CCF is estimated by the $\gamma$ parameter.

To evaluate the strength of the correlation between the estimated RV's and the different stellar activity indicators, we calculated the Pearson correlation coefficient, which in its general form is defined as:
%
\begin{equation}
R (x,y)= \frac{cov(x,y)}{\sigma(x),\sigma(y)},
\label{eq:Pearson:corr}
\end{equation}
%
where $x$ and $y$ are two quantitative variables, $cov(x,y)$ indicates the covariance between $x$ and $y$, and $\sigma(x)$ and $\sigma(y)$ represent their standard deviations.  A $p-$value for the statistical test having null hypothesis $H_{0}: R=0$ is provided, along with a $95\%$ confidence interval for $R$ when needed.

%-----------------------------------------------------------------------------------------------------------------------------------------------
\section{Radial Velocity correction for stellar activity} \label{sec:31}

\xavier{Exoplanets only produce a pure RV signal. Stellar activity on the contrary, and in particular the presence of active regions on the stellar photosphere, does not produce a blueshift or redshift of the entire stellar spectra, but creates a spurious RV signal by modifying the shape of spectral lines. To track these variations in the shape of the spectral lines}, the general approach consists in using the FWHM, the BIS SPAN or the indicators introduced in \citet{Boisse-2011} or \citet{Figueira-2013}, which provide information on the width and asymmetry of the CCF. A strong correlation between the estimated RVs and one or more of these parameters provides an indication that stellar activity signals are affecting our measurements.

\xavier{When fitting for planetary signals in RV data, it is common to include in the fitted model, linear dependencies with the BIS SPAN and the FWHM, to take into account the signal induced by stellar activity \citep[e.g.][]{Dumusque:2017aa,Feng:2017aa}.
%Some people also include the \logrhk, however, \citet{Feng-2018} show that this is probably not a good choice as \logrhk and RV are not well correlated.
In this paper, we propose to add additional parameters in the model to correct for stellar activity: first the amplitude parameter A of the CCF, generally referred to as the CCF contrast, and the interaction between the BIS SPAN and the FWHM (or $\gamma$ and SN FWHM in the SN case). The stellar activity correction we propose can therefore be written as:}
%
\begin{equation}
RV_{\text{stellar activity}}= \beta_{0} + \beta_{1} A + \beta_{2} \gamma + \beta_{3} \text{SN FWHM} + \beta_{4} (\gamma  \text{SN FWHM})+\epsilon,
\label{eq:RV:correction}
\end{equation}
%
where $\beta_{0}$ is the intercept and $\epsilon$ is the vector of the errors with mean equal to $0$ and covariance matrix equal to $\sigma^{2}I$ ($I$ defined as the identity matrix). 
The contrast parameter $A$ accounts for the presence of a spot on the stellar surface, which produces a change in the amplitude of the CCF and not only on its asymmetry or width \citep[see e.g. Fig. 2 in ][]{Dumusque-2014b}.
The reasons for including in Eq.~\ref{eq:RV:correction} a variable that quantifies the interaction between $\gamma$ and SN FWHM (or BIS SPAN and FWHM) will be better understood through the results of the examples presented in Sec.\ref{sec:soap}. 
\xavier{This interaction term can accounts for possible interactions between SN FWHM (or FWHM) and $\gamma$ (or BIS SPAN), meaning that the two variables' association with the response, $RV_{\text{stellar activity}}$, depends also on the other variable.
}. \xavier{Umberto, Jessi, is this correct ?} \jessi{I modified the statement - is the modification clear?}
%while the association between in this case BIS SPAN and FWHM (or $\gamma$ and SN FWHM) means that the values of one variable relate to the values of the other (since in this case we have two quantitative variables we talk about correlation), with the term interaction we mean that the effect that one variable has on the RV's is not constant. In particular the effect differs at different values of the other values. %As a consequence of this, if two variables are interacting the may or may not be associated.

In order to show the goodness of this correction, a statistical test on the parameters $\beta_{0}$, $\beta_{1}$, $\beta_{2}$, $\beta_{3}$ and $\beta_{4}$ is presented, where the null hypothesis is $H_{0}: \beta_{i}=0$, for $i=0,\dots,4$. The level for not rejecting the null hypothesis is fixed to $0.05$. The coefficient of multiple correlation $R^2$ is introduced in order to explain how well this linear combination addresses the variability of the RV variation induced by stellar activity. 

When working with a linear regression, there are several ways to select the variables to include in the model. While usually the stepwise technique is used \citep{efroymson1960multiple,hocking1976biometrics}, the proposed function defined in Eq.~\ref{eq:RV:correction}, that accounts for stellar activity, is the result of statistical and astronomical considerations. In particular we checked that the correlations between the proposed parameters were not approaching one: if it was the case, the matrix needed to calculate the estimates would be singular, hence non invertible. This problem is known in statistics with the term multicollinearity. A detailed discussion of the topic can be found in the book by \citet{belsley1991}. \xavier{It is common to see some correlation between the amplitude parameter $A$ and the FWHM (or SN FWHM) of the CCF. However, in the analysis of real data presented in this work, we never observed a correlation coefficient exceeding $0.66$ and therefore, the problem of multicollinearity is avoided. Finally, we investigate as well the statistical significant of the interaction term between $A$ and the width, and $A$ and the asymmetry of the CC, however, those interaction were relevant for accounting for stellar signal.}

%-----------------------------------------------------------------------------------------------------------------------------------------------
\section{Simulation Study} \label{sec:soap}
In order to evaluate the performance of the proposed SN approach for modelling the CCF and the benefit of using the proposed correction for stellar activity (See Eq.~\ref{eq:RV:correction}), we begin by considering a simulation study using spectra generated from the Spot Oscillation And Planet 2.0 code \citep[SOAP 2.0,][]{Dumusque-2014b}.

\xavier{For a given configuration of spots and faculae on the stellar surface, SOAP 2.0 gives as output the simulated CCF as a function of rotational phase. The code returns as well the RV and the FWHM by fitting a Normal density to the CCF, and the BIS SPAN by calculating the bisector of the CCF. SOAP 2.0 gives us therefore noiseless CCFs affected by stellar activity, which will be used to compare the benefits of fitting a SN density to the CCF compared to a Normal density fit.}

\xavier{For the simulations shown below, we modelled a star similar to the Sun. The stellar rotational period is set to 25.0 days, the radius to a solar radius and the star is seen equator on. The stellar effective temperature is set to 5778 K (NASA Planetary Fact Sheets), and we use a quadratic limb-darkening relation with linear and quadratic coefficients 0.29 and 0.34, respectively \citep[][]{Oshagh-2013a, Claret-2011}. In addition, to be able to compare the result of those simulations with real data obtained with the HARPS spectrograph in Sect.~\ref{sec:4}, we asked SOAP 2.0 to generate CCFs of width 40 \kms and of resolution 115'000, so that those CCFs would have the same properties as the one returned by the HARPS data reduction.}

\subsection{Faculae} \label{sec:soap.faculae}

\xavier{To see the impact of a faculae on the different parameters of the CCF, we simulated the effect of an equatorial faculae of size 3\% relative to the visible stellar hemisphere. The faculae is face on when the phase equals to 0. Note that a 3\% faculae is relatively big for the Sun. At maximum activity, big faculae have generally a size of 1\%.
In Fig.~\ref{fig:faculae}, we compare the barycentric variation of the CCF as measured when fitting a Normal density and taking its mean (RV), and when fitting a SN density and taking its mean (SN mean RV) or its median (SN median RV). We see that all the different estimates of the CCF barycenter present a signal of similar amplitude, however the signal obtained with SN mean RV is slightly different with respect to the two others, with a maximum amplitude happening at a different phase.}

\begin{figure}[htbp]
   \centering
\includegraphics[width=3.6in]{RV_comparison_FACULAE.pdf} 
%\includegraphics[height = 2.5in]{RV_se_comparison_FACULAE.pdf} 
\caption{RV's changes as function of the orbital phase in the case in which a faculae is present on the photosphere of the star. SN mean RV seems to have the smallest spurious variations caused by the faculae.}
   %\caption{(left) RV's changes as function of the orbital phase in the case in which a faculae is present on the photosphere of the star. SN mean RV seems to have the smallest spurious variations caused by the faculae. (right) Evaluation of the standard errors corresponding to the defined RVs. The standard errors retrieved for SN median RV are $10 \%$ smaller than the standard errors derived for RV. SN mean RV has the largest related uncertainties.}
    \label{fig:faculae}
\end{figure}

\xavier{In Fig.~\ref{fig:faculae.corr}, we compare the correlation between the different RV estimates and the different asymmetry or width estimates. As we can see, the strength of the correlation between $\gamma$ and SN mean RV, and $\gamma$ and SN median RV are much stronger than the correlations between BIS SPAN and RV, with Pearson correlation coefficient $R$ values of 0.46, -0.67 and -0.09, respectively. Regarding the width barycenter correlations, we also find a stronger correlation between SN FWHM and SN mean RV compared to the one between FWHM and RV, $R=0.98$ and 0.84, respectively. In this case however, the correlation between SN FWHM and SN median RV is smaller with $R=0.50$. This first analysis shows that in the case of a faculae, using some parameters from the SN can lead to much stronger correlation than the usual Normal parameters and therefore, the SN parameters seems to be better 

probe better stellar activity. We will of course confirm this in the next sections presenting the case of a spot and of a spot plus a planet, but also in Sec~\ref{sec:4} when we will apply the SN fitting to real data sets.}

\begin{figure*}[htbp]
   \centering
\includegraphics[height = 6in]{SOAP_FACULAE_Comparison_para_SN.pdf} 
   \caption{Evaluation of the correlation between the RV's and the asymmetry parameters when a faculae is present on the photosphere of the star. In this case both the shape and the width of the CCF changes as the faculae moves, producing statistically significative correlations between the RV's and respectively the asymmetry parameter and the width parameter.}
    \label{fig:faculae.corr}
\end{figure*}

Since the RV variation displayed in Fig.~\ref{fig:faculae} is caused by only stellar activity, in this case a faculae, \xavier{we applied the activity correction proposed in Eq.~\ref{eq:RV:correction} to check its efficiency in the case of a faculae.} The results of this correction are displayed in Fig.~\ref{fig:faculae.correction} and the statistical tests on the coefficients involved in Eq.~\ref{eq:RV:correction} are summarized in Table \ref{table:faculae.test}. 
It is straightforward to see that the proposed correction for \xavier{stellar activity is able to account for the majority of the activity signal created by a faculae, with a $R^2$ of our model larger than 0.95. In addition, the rms of the different estimates of the RV reduces from about 3 \ms before correction to values below 0.15 \ms after correction.
We see a slightly smaller rms after correction for the SN parameters, however the different is not significant. When comparing the correction proposed in Eq.~\ref{eq:RV:correction} with what is generally used, i.e. a linear combination of the asymmetry and width parameter, we see that the proposed correction is able to reduce the rms of the RV residuals by a factor of 2. Looking at the significance of the coefficients in table \ref{table:faculae.test}, we observe that the \umberto{parameter related to the} intercept, $\beta_0$, is only significant at a level of 1\% in the case of the Normal parameters \umberto{or when SN median RV is used}.}

\begin{figure*}[htbp]
   \centering
\includegraphics[height = 6in]{FACULAE_NEW_CORRECTION_[3]CorrectionActivity_RadialVelocity_vs_time.pdf} 
   \caption{\umberto{Set of spurious variations in RV's caused by a faculae using a Normal and a SN fit before and once corrected from stellar activity. The new correction is done using Eq.~\ref{eq:RV:correction} and the estimated parameters are presented in Table \ref{table:faculae.test}. Once corrected for stellar activity using the proposed linear function (black dots), the residuals present a smaller systematic component if compared with the residuals obtained with the usual correction (blue triangles).}}
    \label{fig:faculae.correction}
\end{figure*}

\begin{table}
\centering
\begin{tabular}{|c|c|c|c|}
\hline
Parameter          & N mean RV         &   SN mean RV &   SN median RV \\
\hline
$\beta_{0}$            &    $0.033$    & $0.00020$ & $0.61$ \\
\hline
$\beta_{1}$            &    $2.22e-16 $    & $2.22e-16 $ & $2.22e-16 $ \\
\hline
$\beta_{2}$            &     $0.0034$   &  $2.22e-16 $ & $2.22e-16 $\\
\hline
$\beta_{3}$            &     $0.00016$   &  $1.091e-6$ & $9.75e-7$\\
\hline
$\beta_{4}$            &     $2.22e-16$   &  $2.22e-16$ & $2.22e-16$\\
\hline
$R^{2}$      &     $0.9978$    &  $0.9985$ & $0.9981$  \\
\hline
\end{tabular}
\caption{\umberto{Evaluation of the linear combination used for correcting the RV's from spurious variations caused by a \textbf{faculae}, according to Eq.~\ref{eq:RV:correction}. All the variables, except the intercept when the independent variable is SN median RV, are statistically useful to explain spurious variations in RV's caused by a faculae only for the SN analyses. The estimated $R^{2}$ show that the proposed correction for stellar activity explains all the spurious variability in RV's.}}
\label{table:faculae.test}
\end{table}

\subsection{Spot} \label{sec:soap.spot}

\xavier{In this section, we simulate the effect on the CCF parameters of an equatorial spot of size 1\% relative to the visible stellar hemisphere. The spot is face on when the phase equals to 0. Note that this is an extremely big spot for the Sun, as in general big spots are more in the regime of 0.1\%.
In Fig.~\ref{fig:spot}, we shows the barycentric variation of the CCF induced by this simulated spot. Contrary to the case of the faculae seen before, for the spots, all the different estimates of the CCF barycenter have exactly the same shape in variation. The amplitude for SN mean RV is however slightly smaller.} 

Fig.~\ref{fig:spot.corr} shows the correlations between the asymmetry parameters and the different estimates for the CCF barycenter: SN mean RV, SN median RV and RV. \xavier{The correlation between $\gamma$ and SN median RV is the strongest with a $R=0.94$, followed then by the correlation BIS SPAN -RV and $\gamma$-SN mean RV, with $R=0.86$. Regarding the correlation between the width and the CCF barycenter, we note that the variation is seen as a circle in this parameter space and therefore no correlation is observed. Once again, like in the case of the faculae, we see that some parameters of the SN gives stronger correlation than when using the Normal parameters.}

\begin{figure}[htbp]
   \centering
\includegraphics[width=3.6in]{RV_comparison_SPOT.pdf} 
%\includegraphics[height = 2.5in]{RV_se_comparison_PLANET_SPOT.pdf} 
\caption{RV's changes as function of the orbital phase in the case in which a spot is present on the photosphere of the star. SN mean RV seems to have the smallest spurious variations caused by the faculae.}
   %\caption{(left) RV's changes as function of the orbital phase in the case in which a spot is present on the photosphere of the star. SN mean RV seems to have the smallest spurious variations caused by the faculae. (right) Evaluation of the standard errors corresponding to the defined RVs. The standard errors retrieved for SN median RV are $10 \%$ smaller than the standard errors derived for RV. SN mean RV has the largest related uncertainties.}
    \label{fig:spot}
\end{figure}

\begin{figure*}[htbp]
   \centering
\includegraphics[height = 6in]{SOAP_SPOT_Comparison_para_SN.pdf} 
   \caption{Evaluation of the correlation between the RV's and the asymmetry parameters when a spot is present on the photosphere of the star. In this case only the shape of the CCF changes as the spot moves, producing statistically significative correlations only between the RV's and the asymmetry parameter.}
    \label{fig:spot.corr}
\end{figure*}

As before, we corrected the originally RV's by using Eq.~\ref{eq:RV:correction}. The results of the correction are displayed in Fig.~\ref{fig:spot.correction}. Also in this case the proposed correction almost completely addresses the issue when considering the SN or Normal parameters, with values of $R^2$ larger than 0.99. \xavier{Looking at Fig.~\ref{fig:spot.correction}, we see that the activity correction proposed here is able to reduce the signal of a spot from a raw RV rms larger than 4.80\ms down to a rms of 0.38 \ms. In this case the RV rms of the residuals obtained in the case of the Normal parameters is smaller, however,} \xavier{REDO WITH FINAL PLOT with a difference of 6\,$mm.s^{-1}$, we cannot say that this difference is significant. When comparing the activity correction proposed in this paper with what is commonly used, i.e only a linear dependance with the width and asymmetry of the CCF, we see that our solution is capable of reducing the RV residual rms by a factor of 3.5, which is even more than the factor 2 found in the case of the faculae.

In terms of the significance of the different parameters in Eq.~\ref{eq:RV:correction}, summmurized in Table \ref{table:spot.test}, \umberto{we observe that the $\beta_0$, and $\beta_3$, corresponding respectively to the intercept and to the width of the CCF}, are \umberto{not helpful} to understand the variation measured in RV's. This is not surprising when looking at the circle shape drawn when plotting the width as a function of the RV in Fig.~\ref{fig:spot.corr}. We see also that the ampltitude parameter, with coefficient $\beta_1$ is only useful to explain the RV variation of the spot in the case of the Normal distribution.} 

\begin{figure*}[htbp]
   \centering
\includegraphics[height = 6in]{Spot_NEW_CORRECTION_[3]CorrectionActivity_RadialVelocity_vs_time.pdf} 
   \caption{\umberto{Set of  spurious variations in RV's caused by a \textbf{spot} using a Normal and a SN fit before and once corrected from stellar activity. The new correction has done using Eq.~\ref{eq:RV:correction} and the estimated parameters are presented in Table \ref{table:faculae.test}. Once corrected for stellar activity using the proposed linear function (black dots), the residuals present a smaller systematic component if compared with the residuals obtained with the usual correction (blue triangles).}}
    \label{fig:spot.correction}
\end{figure*}

\begin{table}
\centering
\begin{tabular}{|c|c|c|c|}
\hline
Parameter          & N mean RV         &   SN mean RV &   SN median RV \\
\hline
$\beta_{0}$            &    $0.4975$    & $0.21$ & $0.21$ \\
\hline
$\beta_{1}$            &    $2e-16$    & $2e-16$ & $2e-16$ \\
\hline
$\beta_{2}$            &     $2e-16$   &  $2e-16$ & $2e-16$\\
\hline
$\beta_{3}$            &     $0.017$   &  $0.13$ & $0.11$\\
\hline
$\beta_{4}$            &     $2e-16$   &  $2e-16$ & $2e-16$\\
\hline
$R^{2}$      &     $0.9959$    &  $0.9936$ & $0.9952$  \\
\hline
\end{tabular}
\caption{\umberto{Evaluation of the linear combination used for correcting the RV's from spurious variations caused by a spot, according to Eq.~\ref{eq:RV:correction}. All the covariates are statistically useful to explain the variability in RV's caused by a spot, except the intercept and the width of the CCF. The estimated $R^{2}$ show that the proposed correction for stellar activity explains all the spurious variability in RV's.}}
\label{table:spot.test}
\end{table}

\subsection{Spot and planet} \label{sec:soap.spot.planet}

The last simulation presented consists in having a planetary signal influencing the CCF, in addition to the 1\% spot modeled before (see Sec.~\ref{sec:soap.spot}). \xavier{The purpose of this example is to check if we are able to disentangle as efficiently the two different sources of variations when using the parameters derived using a Normal, or a SN fit the the CCF. In this case, we inject a planet with an semi-amplitude of 10 \ms, with no eccentricity, and with a period corresponding to 1/3rd of the stellar rotational phase.}

Fig.~\ref{fig:spot.plus.planet} shows the variation observed in the CCF barycenter parameters. Like in the case of the spot, all barycenter indicators show very similar variations, with SN mean RV showing a slightly smaller amplitude.

In Fig.~\ref{fig:spot.plus.planet.corr}, we show the correlation between the different CCF parameters. \xavier{Except of seing smaller correlation than in the case of the spot, due to the fact that the planet induces changes in the CCF barycenter without any change in any of the width or asymmetry parameters, the correlation strengths between the asymmetry parameters and the CCF barycenter are in exactly the same order than in the case of the spot: $\gamma$--SN median RV $R=-0.84$, BIS SPAN-RV $R=-0.78$ and $\gamma$--SN mean RV $R=-0.76$. The variation seen in the width-CCF barycenter space draw a circle like in the case of a spot, therefore, no correlation is observed between those parameters.}

\begin{figure}[htbp]
   \centering
\includegraphics[width = 3.6in]{RV_comparison_SPOT_PLANET.pdf} 
%\includegraphics[height = 2.5in]{RV_se_comparison_SPOT.pdf} 
 \caption{RV's changes as function of the orbital phase in the case in which a spot is present on the photosphere of the star and a planet is injected. N mean RV seems to have the largest variations caused by the combined action of spot and planet.}
   %\caption{(left) RV's changes as function of the orbital phase in the case in which a spot is present on the photosphere of the star and a planet is injected. N mean RV seems to have the largest variations caused by the combined action of spot and planet. (right) Evaluation of the standard errors corresponding to the defined RVs. The standard errors retrieved for SN median RV are $10 \%$ smaller than the standard errors derived for RV. SN mean RV has the largest related uncertainties.}
    \label{fig:spot.plus.planet}
\end{figure}

\begin{figure*}[htbp]
   \centering
\includegraphics[height = 6in]{SOAP_SPOT_PLANET_Comparison_para_SN.pdf} 
   \caption{Evaluation of the correlation between the RV's and the asymmetry parameters when a spot is present on the photosphere of the star and a planet is injected.  In this case only the shape of the CCF changes as the spot moves, producing statistically significative correlations only between the RVs and the asymmetry parameter. The correlations between the RVs and the width parameter of the CCF is weaker than the previous case that considers only the presence of a spot on the photosphere of the star.}
    \label{fig:spot.plus.planet.corr}
\end{figure*}

In order to correct the RV's from the spurious variation caused by the spot, \xavier{we need to add to our model of activity correction, a signal to take into account the RV variation caused by the injected planet. The observed RV can therefore be modeled by a combination of the activity and the planet signal:}
%
\begin{equation}
RV= RV_{\text{stellar activity}} + RV_{\text{planet}},
\label{eq:RV:correction.overall}
\end{equation}
%
\xavier{where $RV_{\text{stellar activity}}$ can be found in Eq.~\ref{eq:RV:correction}, and $RV_{\text{planet}}$, in the case of no eccentricity, can be modelled by the following sinusoidal function:}
%
\begin{equation}
RV_{\text{exoplanet}}= K \sin \left(\frac{2 \pi}{P} (t - t_{0})\right),
\label{eq:RV:correction.planet}
\end{equation}
%
where the amplitude $K$, the orbital period $P$ and the epoch at the periapsis $t_{0}$ are three unknown parameters that define the planetary orbit.

%We note that the p--value associated with the amplitude parameter $K$ is particularly relevant for rejecting or not rejecting the assumption about the presence of an orbiting companion. Moreover we note also that Eq.~\ref{eq:RV:correction.planet} is highly non linear, meaning that the estimation of all the parameters involved in Eq.~\ref{eq:RV:correction.overall} has to be done numerically. %We used non linear least squares and the results are displayed in Fig.~\ref{fig:spotplanet.correction}. We can see how in this case we are able to disentangle the spurious variations in RV's caused by stellar activity from the pure Doppler-shift due to the planet. 
The statistical tests conducted on the parameters, whose results are summarized in Table \ref{table:spotplanet.test}, \xavier{shows that except for the width parameters with coefficient $\beta_3$, all the other are significantly useful to explain the RV variation induce by a spot plus a planet.}
%\begin{figure*}[htbp]
%   \centering
%\includegraphics[height = 4in]{SPOT_PLANET_NEW_CORRECTION.pdf} 
%   \caption{Set of  variations in RV's estimated using a Normal and a SN fit before and once corrected from stellar activity. In this case there are spurious variations caused by the spot and pure Doppler-shift due to the planet. The correction is done using Eq.~\ref{eq:RV:correction.overall} and the estimated parameters are presented in Table \ref{table:spotplanet.test}. In this case, by solving Eq.~\ref{eq:RV:correction.overall}, we are able to completely disentangle the spurious variations in RV's caused by the presence of the spot from the pure dopplershift caused by the exoplanet.}
%    \label{fig:spotplanet.correction}
%\end{figure*}

\begin{table}
\centering
\begin{tabular}{|c|c|c|c|}
\hline
Parameter          & N mean RV         &   SN mean RV &   SN median RV \\
\hline
$\beta_{0}$            &    $0.00063$    & $2e-16$  & $1.42e-09$ \\
\hline
$\beta_{1}$            &    $2e-16$    & $2e-16$  & $2e-16$ \\
\hline
$\beta_{2}$            &     $2e-16$   & $2e-16$ & $2e-16$\\
\hline
$\beta_{3}$            &     $0.067$   &  $0.40$  & $0.38$\\
\hline
$\beta_{4}$            &     $2e-16$   &  $2e-16$ & $2e-16$\\
\hline
K            &     $2e-16$   &  $2e-16$   & $2e-16$ \\
\hline
P            &     $2e-16$   &  $2e-16$ & $2e-16$ \\
\hline
$t_{0}$            &     $2e-16$   &  $2e-16$ & $2e-16$ \\
\hline
$\text{Residuals}$      &     $0.71 \ms$    &  $ 0.66 \ms$ & $0.70 \ms$  \\
\hline
\end{tabular}
\caption{Evaluation of the linear combination used for correcting the RV's, according to Eq.~\ref{eq:RV:correction.overall}. All the parameters are statistically helpful to address spurious variations in RV's except the FWHM. Concerning the keplerian parameters, the amplitude $K$ that provides relevance about the possibly presence of the exoplanet. Note that since non linear least squares are required, the residual standard error rather than the $R^2$ is displayed for each case.}
\label{table:spotplanet.test}
\end{table}

\subsection{Conclusions on the simulation study} \label{sec:soap.conclusions}

In this Sec.\ref{sec:soap}, we presented a first implementation of the SN fit to the CCF, using SOAP 2.0 to simulate noiseless CCF affected by stellar activity variation. 

\umberto{Do we need to point out the following: \\ If we look at the $R^2$, the proposed function that corrects from stellar activity addresses (almost) all the spurious variations caused in RV's by active regions. Anyway, if we look the residuals, a systematic component is still present, although smaller respect the residual obtained with the usual correction. The residuals are supposed to be homoscedastic with 0 mean under the assumption that the model is correct. Therefore, since our $R^2$ is always close to $1$, we can argue, as we discussed already many times, that the CCF does not follow a normal fit, neither a SN one, otherwise, given our extremely high $R^2$ we should have gotten homoscedastic 0 mean residuals.}

\xavier{is all this discussion really usefull ? Before moving to real cases, where the analyses on five stars are presented, we need to provide further considerations. First of all, looking ad the analyses conducted with SOAP 2.0, it seems that the largest correlation between an asymmetry parameter and a set of RV's happens to be when respectively $\gamma$ and SN median RV are used. This is a bit surprising, since as the shape of the CCF changes, we expect SN median RV to be more robust than SN mean RV. \umberto{A possible justification of this ...}. As second, when searching for stellar activity by deriving the correlation between the set of RV's and either an asymmetry parameter or the width of the CCF, the latter leads to weaker and hence less conclusive results if the active region is a spot. When stellar activity is dominated by faculae, both the shape and the width of the CCF changes as the faculae evolves on the photosphere of the star. Related to these last two considerations, we note that the interaction between the asymmetry and the width of the CCF is useful to explain part of the variability in the RV's if the active region is a spot but not when it is a faculae. The proposed function to correct for stellar activity addressed high level of spurious variations in RV's caused by active regions. In particular, respect to other common linear interpolation, we proposed to use as covariates also the amplitude parameter of the CCF and the interaction between $\gamma$ and SN FWHM (or BIS SPAN and FWHM). As a consequence of using the interaction between the asymmetry and the width of the CCF, we note that the FWHM (or SN FWHM) becomes statistically not significant, while this is not the case if the interaction term is not involved in the linear regression. Finally, the correlations involving the common indicators (i.e. RV, FWHM and BIS SPAN) are systematically weaker than the correlations obtained by fitting the SN to the CCF, suggesting that this density could be helpful when searching for active regions. We recall moreover that all the quantities needed for conducting the analyses of the CCF are directly available by just fitting the SN.}

%In Sec. \ref{sec:5} we will present the evaluation of the standard errors associated with the parameters estimated for all the three presented cases. This step will point out how SN median RV minimizes the uncertainties respect using both SN mean RV and RV, therefore suggesting its use in order to properly define the set of RVs of the star.

%-----------------------------------------------------------------------------------------------------------------------------------------------
\section{Real data application} \label{sec:4}

In this Section we present the analyses conducted on Alpha Centauri b, comparing the result of fitting a CCF using the SN density defined in Sec. \ref{sec:3} with the approach based on fitting a Normal density to retrieve the RV and width of the CCF and calculating the bisector to derive the asymmetry parameter BIS SPAN. Four other stars have been analyzed with the proposed method and details can be found in the Appendix \ref{appendix}. For all the stars that have been considered in the present work, \xavier{we selected CCFs  that were derived from spectra that had a SNR at 550 nm larger than 10.}

% A comparison with the results obtained by the classic approach is done, where the RV is estimated by retrieving the mean of the Normal density used to fit the CCF, along with the FWHM of the Normal density and the BIS SPAN or the other asymmetric parameters defined in \citet{Figueira-2013}. The latter parameters are calculated separately from the Normal fit that leads to the set of RV's of the star.

\subsection{Comparison between the different CCF parameters derived with the Normal and the Skew Normal density} \label{sec:alphacentb}

We analysed the $1808$ CCFs derived from spectra of Alpha Centauri B taken in 2010 by the HARPS spectrograph. Note that more observations have been done during this year, however we selected here only the data that are not significantly affected by contamination from Alpha Cen A \citep[see][]{Dumusque-2012}. \xavier{We choose these observations as their represent probably the best sampled and most precise RV data set showing a strong solar-like activity signal \citep{Thompson-2017, Dumusque-2012}.}

We begin the analyses by evaluating the correlation between $\gamma$ and the BIS SPAN. 
In the left panel of Fig.~\ref{fig:alphacent:corr.gamma}, we see that the relationship between $\gamma$ and the BIS SPAN is linear, with a slope equal to $0.72$ and a strong Pearson correlation coefficient of $R=0.95$. This strong correlation shows that $\gamma$ and BIS SPAN are measuring in a very similar way the CCF asymmetry. \xavier{This strong correlation allows as well to convert the dimensionless $\gamma$ parameters in \ms to compare it with BIS SPAN using the slope of the correlation, in this case 720 \ms.}
%\xavier{As $\gamma$ has no units, this will allow us to compare the amplitude of the activity signal seen in $\gamma$ and in BIS SPAN, by using the slope of this correlation as a scaling factor}.

%
\begin{figure*}[htbp]
   \centering
\includegraphics[height = 3.55in]{HD12862_[2]gamma_vs_bisspan.pdf} 
\includegraphics[height = 3.65in]{HD12862_[2]RadialVelocityDifferences.pdf} 
   \caption{\emph{Left: }Correlation between $\gamma$ and the BIS SPAN for Alpha Centauri B. As we can see with the strong correlation, those two parameters measure the CCF asymmetry in the same way. \emph{Top right:} RVs as function of Julian Day for Alpha Centauri b in 2010. The RVs are retrieved using the mean of a Normal fitted to the CCF (red triangles), or the mean (black circles) or median (cyan crosses) of a SN density fitted to the CCF. \emph{Bottom right:} RV differences between the RV derived when using the SN density and the RV derived when using the Normal density.}
   \label{fig:alphacent:corr.gamma}
\end{figure*}
%

In the right of Fig.~\ref{fig:alphacent:corr.gamma}, we show the comparison between the RVs retrieved using the SN density and the ones obtained with the Normal density. \xavier{We clearly see that the amplitude of the activity signal is stronger when using SN mean RV, and the signal measured using N mean RV or SN median RV are very similar. We observe the same behavior in the case of a faculae (see Sec.~\ref{sec:soap.faculae}) and therefore it seems that the activity signal seen in those data are more likely affected by faculae. This is an observation that was already made in \citet{Dumusque-2014c}.}
%When using SN mean RV, it is possible to observe more variations than the ones measured by the Normal fitting. This happens because the mean of the SN is more sensitive to stellar activity. In fact, because the SN includes an asymmetry parameter, SN mean RV gets more shifted in the direction of the asymmetry induced by stellar activity. On the other hand, when using SN median RV, smaller variations in RV are caused by changes in the asymmetry of the CCF, because this second location parameter is a more robust indicator than the mean. The bottom plot of Fig.~\ref{fig:alphacent:diff:RV} captures this aspect. Both indicators can be used to capture and summarise the different information available in the CCF, as will be shown in the remainder of this work.
%%
%\begin{figure*}[htbp]
%   \centering
%\includegraphics[height = 3in]{HD12862_[2]RadialVelocityDifferences.pdf} 
%   \caption{(top) RV's as function of Julian Day for Alpha Centauri b. The RV's are retrieved using the mean of the Normal (red triangles), SN mean RV (black circles), SN median RV (cyan crosses). (bottom) RV differences between Normal RV and SN mean RV (black circles) and between Normal RV and a SN median RV (cyan crosses).}
%   \label{fig:alphacent:diff:RV}
%\end{figure*}
%

\xavier{Similar to what is done in Sec.~\ref{sec:soap}, we compare the correlation between the asymmetry or the width parameters of the CCF and the RV, i.e. the CCF barycenter, in Fig.~\ref{fig:alphacent:corrPlot}. In addition to what is done in Sec.~\ref{sec:soap}, we add the asymmetry parameters derived in \citet{Boisse-2011}, $V_{span}$ and in \citet{Figueira-2013}, BIS-, BIS+, Bi Gauss and $V_{asy}$, as these authors found those asymmetry parameters more correlated to the RVs than BIS SPAN. It is clear in the case of Alpha Cen B, but also in the four other stars presented in Appendix \ref{appendix}, that the correlation found between $\gamma$ and SN mean RV is always the strongest. On alpha Cen B, the Pearson correlation coefficient reaches a value of $R=0.74$, while it reaches at best $R=-0.42$ for all the other asymmetry-RV correlations not derived using the SN density fit. When looking at the width-RV correlation, once again, the correlation between SN FWHM and SN mean RV is the strongest for all the stars, except for the quietest star of all five, HD10700, for which the Pearson correlation coefficient 0.53 for the Normal parameters and 0.42 for the SN parameters. This analysis shows that the parameters derived when using a SN density are more sensitive the stellar activity, and therefore using those parameters can lead to the detection of stellar activity, when the Normal parameters do not detect anything (see the Appendix \ref{appendix} the case of the the asymmetry-RV correlation for HD10700, HD215152, Corot-7, and the width-RV correlation for HD215152)}

%The correlations between $\gamma$ and SN mean RV, and also SN FWHM and SN mean RV

%Because the median is a more robust index than the mean, the correlation between $\gamma$ and SN median RV is not as large as the correlation between $\gamma$ and SN mean RV, but it is nonetheless $1.5$ times larger than the correlation between the other common asymmetry parameters and their corresponding RV's. In other words, changes in the asymmetry of the CCF are better captured when using the SN mean RV. The correlation between FWHM and the RV's, either by using SN mean RV or SN median RV, is as well stronger when fitting a SN density rather than a Normal. All the correlations are statistically different from $0$. Recalling the analyses presented in Sec. \ref{sec:soap}, we could infer that Alpha Centauri b is dominated by faculae, because the correlations between the RV's and the width of the CCF are strong (in particular the correlation between SN mean RV and SN FWHM is $0.817$).
%%
\begin{figure*}[htbp]
   \centering
\includegraphics[height = 6in]{HD12862_[4]Comparison_para.pdf}  
   \caption{Correlation between the asymmetry parameters and the RVs for Alpha Centauri B. The last three plots show the correlation between the FWHM and the RVs for Alpha Centauri B using respectively the SN (SN mean RV and SN median RV) and the Normal fits. The correlations are always stronger when using parameters derived from the SN fit than the Normal one. The p--values associated with each $R$ is statistically different from $0$.} 
   \label{fig:alphacent:corrPlot}
\end{figure*}
%

\xavier{In Fig \ref{fig:alphacent:correctionRV}, we study the efficiency of the stellar activity correction proposed in Sec.~\ref{sec:31}. The first observation is that the RV measured with SN mean RV presents a rms 35\% larger than the RV measured when fitting a Normal density. In the case of SN median RV, we see as well a larger RV rms, however this one is only 9\% larger. Even though we see those differences in RV, once we correct for the stellar activity using Eq.~\ref{eq:RV:correction}, we find the same RV residuals rms. In the best case, for SN mean RV, we reduce the stellar activity signal by a factor of 2, while in the worst case, N mean RV, only a factor 1.5. Therefore, although it seems that the parameters derived using the SN density are more sensitive to stellar activity, we are not able to correct better for stellar activity signal using a linear combination of the different CCF parameters.}

\xavier{When looking at the significance of the different parameters in the activity correction in Table~\ref{table:alphacent.test}, we see that the intercept is not significant for any of the correction, and BIS SPAN, with coefficient $\beta_2$, is not useful to explain stellar activity when using the parameters derived from the Normal density fit. However all the other parameters in the Normal and SN case are useful. By analyzing the value for $R^2$, we see that the correction is more efficient at explaining stellar activity when considering for the CCF barycenter SN mean RV. This is expected from the results above since for the three different cases, we arrive to the same RV residual rms after correction, but before correction SN mean RV shows the largest RV rms.}

\begin{figure*} 
   \centering
\includegraphics[height = 6in]{NEW_CORRECTIONHD12862_[3]CorrectionActivity_RadialVelocity_vs_time.pdf} 
   \caption{Set of RV's for Alpha Centauri B using a Normal and a SN fit before and once corrected from stellar activity. The correction is done using Eq. \ref{eq:RV:correction}. Once corrected for stellar activity, the residuals in the Normal and SN analyses are comparable.}
   \label{fig:alphacent:correctionRV}
\end{figure*}

\begin{table}
\centering
\begin{tabular}{|c|c|c|c|}
\hline
Parameter          & RV         &   SN mean RV &   SN median RV \\
\hline
$\beta_{0}$            &    $0.49$    & $0.90 $  & $0.027$ \\
\hline
$\beta_{1}$            &    $2.22e-16$    & $2.22e-16 $  & $2.22e-16$ \\
\hline
$\beta_{2}$            &     $0.33$   & $2.22e-16 $ & $1.23e-11$\\
\hline
$\beta_{3}$            &     $ 2.22e-16$   &  $2.22e-16 $  & $ 2.22e-16$\\
\hline
$\beta_{4}$            &     $2.22e-16$   &  $2.22e-16 $ & $ 2.22e-16 $\\
\hline
$R^{2}$      &     $0.57$    &  $0.78$ & $0.66$  \\
\hline
\end{tabular}
\caption{Evaluation of the linear combination used for correcting the RV's, according to Eq.~\ref{eq:RV:correction}. Concerning the Normal fit, all the parameters but the intercept and the BIS SPAN are useful in explaining variations in RV's of the star that can be caused by stellar activity. For the SN fit we note that the parameter related to $\gamma$ is highly significant to address part of the spurious variations in RV's caused by stellar activity. The evaluation of the $R^2$ shows that the proposed linear combination better explains variations in RV's due to stellar activity coming from the SN analysis that uses SN mean RV.}
\label{table:alphacent.test}
\end{table}

%Both the proposed indicators coming from the SN density have advantages and limits: SN mean RV better catches changes in the asymmetry of the CCF but the resulting set of RV's ends up being contaminated by those spurious shifts caused by stellar activity that have been shortly presented in Sec. \ref{intro}. When using SN median RV, the final set of RV's is less affected by those spurious shifts caused by stellar activity, but at the same time this indicator is not able to catch as well as SN mean RV changes in the shape and in the width of the CCF. Once corrected from stellar activity using Eq.~\ref{eq:RV:correction}, the results are comparable. Anyway, both SN mean RV and SN median RV are useful to catch different aspects of the CCF and our suggestion is to use SN mean RV when interested in retrieving information about changes in the shape and/or the width of the CCF. In order to provide a set of RV's containing the smallest amount of spurious contamination imputable to stellar activity (i.e. before to run Eq.~\ref{eq:RV:correction}), our suggestion is to use instead SN median RV. 

%-----------------------------------------------------------------------------------------------------------------------------------------------
\subsection{Detection limits when using the RV derived with the Normal or the Skew Normal density} \label{sec:soap_real}

\xavier{In the preceding section, we saw that the RV measured when considering a SN or a Normal density present different amplitudes, mainly in the case of SN mean RV. However, when correcting for stellar activity using the linear combination presented in Eq.~\ref{eq:RV:correction}, it seems that we are able to reach the same RV precision. In this section, we want to investigate if for some of the three different RV definitions presented in this paper, it is easier or more difficult to detect planetary signals when considering stellar activity. To do so, we estimated what would be the minimum amplitude of a planetray signal we could detect at different orbital periods.

To get CCF affected by realistic stellar activity signals, we took the ones from Alpha Cen B used in the preceding section. To simulate a planetary signal, when then blue- or red-shifted those CCF with the wanted amplitude, period and phase.

We first simulated several RV data sets, with always the same stellar signal, but injecting planets with parameters corresponding to the following grid:
\begin{itemize}
\item period of 3, 5, 7, 9, 11, 15, 20 , 25 and 30 days,
\item amplitude from 0.5 to 3 \ms\, by steps of 0.05\,\ms,
\item 10 different phases, evenly sampled between 0 and 2$\pi$.
\end{itemize}

For each simulation, 4500 in total, we applied the activity correction presented in Eq.~\ref{eq:RV:correction} and look for signals in the residuals using a Generalized Lomb-Scargle periodogram \citep[GLS][]{Zechmeister-2009}. If a signal with a p--value\footnote{estimated by bootstrap} smaller than 1\% had a period compatible with the injected planetary period considering an error of 20\%, we were estimating this signal as significant, and therefore the corresponding planet as detected.

Finally, for each period, we searched for the minimum amplitude for which at least 50\% of the planets with different phases were detected. The results are shown in Fig.~\ref{fig:detection_limits.pdf}. As we can see, all the different RV estimates gives extremely similar detection limits. Therefore, we can use any of these estimates when searching for planetary signal in RV data contaminated by stellar activity.

Although the detection limits stay constant for nearly all the periods, we see that for 30 days, we have a huge increase for all the RV estimates. This is because the planet period is close to the rotation period of the star \citep[36.7 days,][]{Dumusque-2012}, and therefore close to the semi-periodicity of the stellar activity signal. In such a case, the activity correction absorbs part of the planetary signal. We thus need to inject a much stronger planetary signal at this period if we want to detect it.}


%-----------------------------------------------------------------------------------------------------------------------------------------------
\section{Estimation of standard errors for the CCF parameters} \label{sec:5}

\xavier{In this section, Because a CCF is obtained from a cross-correlation, each point in a CCF is correlated with each other. Therefore, we cannot simply shuffle each point in the CCF with their respective error bar and then recalculate the best SN or Normal density fit, to see how the noise of the CCF influence the noise on the parameters derived from the CCF. We therefore decided to start with the spectrum, for which we know that each points are totally independent of the others. The standard error on each point is given by the photon noise, which follows a Poisson law and is therefore the square root of the measured flux.}

In this Section, we perform a bootstrap analysis \citep[e.g.][]{davison1997bootstrap, efron1994introduction} in order to retrieve the standard errors associated to SN mean RV, SN median RV, RV, SN FWHM, FWHM, $\gamma$ and BIS SPAN. Because a CCF is obtained from a cross-correlation, each point in a CCF is correlated with each other. Therefore, we cannot do a bootstrap analysis on perturbing independently each CCF point with a Normal density scaled to the error of each given point. A detailed discussions of the methods nowadays available to resampling in situations with dependent data structures is available in \citet{lahiri2013resampling}. All the bootstrap methods that deal with dependant data structures rely on the so called Block Bootstrap methods, originally introduced by \citet{wilks1997resampling}. In our particular case, since each point in a CCF is correlated with each other, we bootstrap a hundred times the stellar spectrum given the photon-noise error of each wavelength and calculate for each realization a new CCF. We then fit a Normal or a SN density to each of these CCF's and then calculate the standard deviations of the density for the location parameters (RV, SN mean RV or SN median RV), the width parameters (SN FWHM or FWHM ) and the asymmetry parameters ($\gamma$ or BIS SPAN).

\subsection{Estimation of standard errors for the CCF parameters for the simulation study} \label{sec:bootstrap_soap}

We start by calculating the standard errors of the parameters retrieved in Sec. \ref{sec:soap}, where with SOAP we produced CCF's presenting spurious variations in RV's caused by a faculae or by a spot. In the third and final case we considered, beyond the spot, a planetary signal that produces pure Doppler-shifts in the CCF's.

Fig.~\ref{fig:se.soap.faculae} shows the results of the bootstrap analysis performed when a faculae is present on the photosphere of the star. The series of three plots in the top of Fig.~\ref{fig:se.soap.faculae} shows the different errors for the RV's, defined as RV (red triangles), SN mean RV (black circles) or SN median RV (cyan crosses), the width of the CCF, defined as FWHM (red triangles) or SN FWHM (black circles) and the asymmetry of the CCF, defined as $\gamma$ (black circles) or BIS SPAN (red triangles). In the series of three plots in the bottom of Fig.~\ref{fig:se.soap.faculae} we show the ratio between errors associated to the parameters derived from the bootstrap analysis fitting the SN and the errors associated to the parameters derived from the bootstrap analysis fitting the Normal density.  We used the same notation also for the other two cases, where respectively a spot is present on the photosphere of the star (Fig.~\ref{fig:se.soap.spot}) and a spot and a planet are introducing both spurious and pure variations in the RV's (Fig.~\ref{fig:se.soap.spot.planet}). Concerning the standard errors related to the RV's if a faculae is creating spurious signals, the ratio between the RV error measured by the bootstrap using the SN and Normal fitting is $1.4$, when using SN mean RV and $0.8$ when using SN median RV. By using SN median RV we get standard errors $20\%$ smaller than using the Normal fit and its corresponding mean. Regarding the errors in width of the CCF, we see that the bootstrap analysis for the Normal and the SN are comparable. Therefore, the precision in the width of the CCF is the comparable if we fit a Normal or a SN to the CCF. Finally, for the errors in evaluating the asymmetry of the CCF, we see that, when fitting the SN to the CCF, the asymmetry errors are $20\%$ smaller. Therefore, the SN fit gives a better precision in CCF asymmetry than what can be reached using BIS SPAN.

\begin{figure*}[htbp]
   \centering
\includegraphics[height = 6in]{RV_comparison_FACULAE_standard_errors.pdf} 
   \caption{Faculae Case. Comparison between the standard errors using the bootstrap analysis for the RV's, the FWHM and the asymmetry parameter. When using SN mean RV (black circles), the standard errors are in average $40\%$ larger than the standard errors retrieved fitting a Normal (red triangles). However, if using SN median RV (cyan crosses), the standard errors are in average $20\%$ smaller than the standard errors coming from the Normal fit. To use as asymmetry parameter $\gamma$ of the SN leads to standard errors in average $20\%$ smaller than the standard errors related to the BIS SPAN. \umberto{explain what happens for those CCF 15 to 19 where s.e. decrease.} Note that for the asymmetry, the error in BIS SPAN is in \kms. To be able to compare the errors in $\gamma$ and BIS SPAN, we multiplied the error in $\gamma$ by the slope of the correlation between $\gamma$ and BIS SPAN.}
   \label{fig:se.soap.faculae}
\end{figure*}

Fig.~\ref{fig:se.soap.spot} shows the results of the bootstrap analysis performed when a spot is present on the photosphere of the star. The series of plots follows the specifications outlined for the previous case. Concerning the standard errors related to the RV's, the ratio between the RV error measured by the bootstrap using the SN and Normal fitting is $1.3$ when using SN mean RV and $0.8$ when using SN median RV. Regarding the errors in width of the CCF, we see that the bootstrap analysis for the Normal and the SN are comparable. Therefore, the precision in the width of the CCF is the comparable if we fit a Normal or a SN to the CCF. Finally, for the errors in evaluating the asymmetry of the CCF, we see that, when fitting the SN to the CCF, the asymmetry errors are $20\%$ smaller.

\begin{figure*}[htbp]
   \centering
\includegraphics[height = 6in]{RV_comparison_SPOT_standard_errors.pdf} 
   \caption{Spot case. Comparison between the standard errors using the bootstrap analysis for the RV's, the FWHM and the asymmetry parameter. When using SN mean RV (black circles), the standard errors are in average $30\%$ larger than the standard errors retrieved fitting a Normal (red triangles). However, if using SN median RV (cyan crosses), the standard errors are in average $20\%$ smaller than the standard errors coming from the Normal fit. To use as asymmetry parameter $\gamma$ of the SN leads to standard errors in average $20\%$ smaller than the standard errors related to the BIS SPAN. Note that for the asymmetry, the error in BIS SPAN is in \kms. To be able to compare the errors in $\gamma$ and BIS SPAN, we multiplied the error in $\gamma$ by the slope of the correlation between $\gamma$ and BIS SPAN.}
   \label{fig:se.soap.spot}
\end{figure*}

Fig.~\ref{fig:se.soap.spot.planet} shows the results of the bootstrap analysis performed when a spot is present on the photosphere of the star. The series of plots follows the specifications outlined for the previous two cases. The conclusions are comparable to the case in which only a spot is present on the photosphere of the star. The ratio between the RV error measured by the bootstrap using the SN and Normal fitting is $1.3$ when using SN mean RV and $0.8$ when using SN median RV. The errors in width of the CCF are comparable and the errors in evaluating the asymmetry of the CCF are $15\%$ smaller when using the asymmetry parameter $\gamma$ of the SN.

\begin{figure*}[htbp]
   \centering
\includegraphics[height = 6in]{RV_comparison_SPOT_PLANET_standard_errors.pdf} 
   \caption{Spot and Planet case. Comparison between the standard errors using the bootstrap analysis for the RVs, the FWHM and the asymmetry parameter. When using SN mean RV (black circles), the standard errors are in average $30\%$ larger than the standard errors retrieved fitting a Normal (red triangles). However, if using SN median RV (cyan crosses), the standard errors are in average $20\%$ smaller than the standard errors coming from the Normal fit. To use as asymmetry parameter $\gamma$ of the SN leads to standard errors in average $15\%$ smaller than the standard errors related to the BIS SPAN. Note that for the asymmetry, the error in BIS SPAN is in \kms. To be able to compare the errors in $\gamma$ and BIS SPAN, we multiplied the error in $\gamma$ by the slope of the correlation between $\gamma$ and BIS SPAN.}
   \label{fig:se.soap.spot.planet}
\end{figure*}

\subsection{Estimation of standard errors for the CCF parameters for real stars} \label{sec:bootstrap_real_star}
In the top plots of Fig.~\ref{fig:se} we show the different errors for the RVs, either defined as RV (red triangles), SN mean RV (black circles) or SN median RV (cyan crosses), the width and the asymmetry of the CCFs for three star, HD215152, HD192310 and Corot-7, that are all at different SNR levels. The parameter SN50 corresponds to the SNR in order 50, which defines a wavelength of 550\,nm. In the bottom plots, we show the ratio between the parameters derived from the bootstrap analysis fitting the SN and the parameters derived from the bootstrap analysis fitting the Normal density. We first see that the errors on the CCF parameters only depends on the SNR and do not depend on the spectral type. This is true if the spectral type are not too different though, like here where we show the results for G and K dwarfs.

Concerning the standard errors related to the RVs, the ratio between the RV error measured by the bootstrap using the SN and Normal fitting is $1.6$ when using SN mean RV and $0.9$ when using SN median RV. In other words, by using SN median RV as parameter that defines the radial velocity of the star given a CCF, we get standard errors $10\%$ smaller than using the Normal fit and its corresponding mean. This result is consistent with what we observed with the simulation from SOAP presented in Sec. \ref{sec:soap}.

Regarding the errors in width of the CCF, we see that the bootstrap analysis for the Normal and the SN are comparable. Therefore, the precision in the width of the CCF is the comparable if we fit a Normal or a SN to the CCF.

Finally, for the errors in evaluating the asymmetry of the CCF, we see that, when fitting the SN to the CCF, the asymmetry errors are $15\%$ smaller. Therefore, the SN fit gives a better precision in CCF asymmetry than what can be reached using BIS SPAN. We recall moreover that, using the SN, all parameters are automatically retrieved in 1 single step, while in the common approach the RV and the BIS SPAN are calculated separately.

%
\begin{figure*}[htbp]
   \centering
\includegraphics[height = 6in]{[5]Errors_vs_SNR_all_stars.pdf} 
   \caption{Comparison between the standard errors using the bootstrap analysis for the RVs, the FWHM and the asymmetry parameter. When using SN mean RV (black circles), the standard errors are in average $60\%$ larger than the standard errors retrieved fitting a Normal (red triangles). However, if using SN median RV (cyan crosses), the standard errors are in average $10\%$ smaller than the standard errors coming from the Normal fit. To use as asymmetry parameter $\gamma$ of the SN leads to standard errors in average $15\%$ smaller than the standard errors related to the BIS SPAN. Note that for the asymmetry, the error in BIS SPAN is in \kms. To be able to compare the errors in $\gamma$ and BIS SPAN, we multiplied the error in $\gamma$ by the slope of the correlation between $\gamma$ and BIS SPAN.}
   \label{fig:se}
\end{figure*}

%, using the spectrum derivative
%, the pure photon-noise error on RV measured 
%Together with the pointwise estimates, retrieving the standard errors for the RVs of the star, $\gamma$ and the FWHM provides information about the uncertainties associated to each parameter. Since the procedure for retrieving the RVs of the star consists in fitting with a density function a CCF, getting the standard errors for all the parameter of interests cannot be done by using the common asymptotic theory. The implementation of a standard bootstrap analysis cannot as well be done, because the points of the CCF are highly correlated. In the analysis of the present work, in order to get the standard errors for the parameters of interest for a particular CCF, we moved each point of this CCF according to its measurement errors, producing $100$ simulated CCFs from the original one. Then we run both the Normal and the SN analyses on these simulated CCFs, letting us to retrieve an evaluation of the standard errors. This operation has been repeated for all the CCFs available for the analyzed stars. The comparison between the standard errors retrieved in this way with the classic noise parameter, {\color{red} obtained by using the Gray's equation cit. Gray (1983)} are presented in Fig.~ \ref{fig:se}.  {\color{red} Xavier: Is it correct how the noise parameter is obtained? Or it is simply $\sqrt(photons)$?}

%Given the strong correlation between the $\gamma$ parameter of the SN and its mean, the standard errors are larger respect the Normal fitting analysis. This fact confirms how correcting the RVs for stellar activity using Eq. \ref{eq:RV:correction} is more helpful when using the SN fitting. The standard errors for the FWHM are comparable for both the SN and the Normal analyses, whereas an improvement is clearly measurable for the standard errors associated to the asymmetry indicators. The last plot in Fig.~\ref{fig:se} shows that the standard errors for $\gamma$ are $10 \%$ smaller than the ones retrieved for the BIS SPAN.

%-----------------------------------------------------------------------------------------------------------------------------------------------
\section{Discussion} \label{sec:discu}

An analysis of the CCF residuals after fitting a Normal or SN density shows that the SN is a slightly better model to explain the shape of the CCF. This comes from the fact that CCF's present a natural asymmetry due the convective blueshift.

We tested at first our assumptions by using simulated CCF's retrieved using the software SOAP 2.0. We then compared for five real stars the difference between the RV's (defined as mean of a Normal, mean of the SN or median of the SN), FWHM and asymmetry (BIS SPAN in the Normal case and $\gamma$ in the SN case). The $\gamma$ parameter is linearly dependent on the BIS SPAN, with always a strong correlation coefficient. 
%To compare $\gamma$, that do not have any units, with BIS SPAN expressed in \kms, one has simply to multiply $\gamma$ by the slope of the linear dependence between $\gamma$ and BIS SPAN. 
The slope of this linear correlation changes depending on the studied star. This is probably because the spectral type is different, therefore the effects from stellar activity are different.

When using as parameter for the RV the mean of the SN, the standard errors are in average $60\%$ larger than the standard errors retrieved fitting a Normal. However, once the RV is defined as the median of the SN (cyan crosses), the standard errors are in average $10\%$ smaller than the standard errors coming from the Normal fit. When looking at the correlation between the asymmetry and width parameters of the CCF (FWHM and BIS SPAN or the alternative indicators in \citet{Figueira-2013} in the Normal case, and SN FWHM and $\gamma$ in the SN case) with respect to the RV's (RV's in the Normal case or SN RV's in the SN case), we observe that the correlations are always stronger for the parameters of the SN. Therefore, the SN parameters are more sensitive to activity. In the case of Tau Ceti, which is at very low activity level, we find a significant correlation of $0.322$ between $\gamma$ and SN mean RV, while for all the other asymmetric parameterization, BIS SPAN or the alternative indicators in \citet{Figueira-2013}, the correlations are weaker with a maximum of 0.$225$.

%%%%%%%%%%%%%%%%%%%%%%%%%%%%%%%%%%%%%%%%%%%%%%%%%
\section{Conclusion} \label{sec:conclu}

In this paper we introduced a novel approach based on the Skew Normal (SN) density to retrieve RV's and shape variations in the CCF of stars. When searching for small-mass exoplanets using the RV technique, it is essential to understand variation of the shape of the CCF, which is a proxy for stellar activity effects. The standard approach consist at first to fit a Normal density to the CCF to get the RV and FWHM, defined as the mean and the FWHM of the Normal density, and then to measure the asymmetry of the CCF by calculating the BIS SPAN. FWHM and BIS SPAN give us information on the line shape that are used to probe stellar activity signals. 

In this paper we propose to conduct the analysis fitting a SN density to the CCF. Since the CCF presents a natural asymmetry due the convective blueshift, the SN density can in principle better catch these aspects respect the Normal fit. On top of that, by using the SN density to fit CCF's, we can measure simultaneously the RV of the star, the width and the asymmetry of the CCF.

Starting from the simulation environment SOAP and then moving to real stars, we showed that using the SN density to fit CCF's leads to a significant improvement to probing stellar activity. While for the Normal density mean and median are equivalent, using the SN fit different location parameters can be tested. While the median of the SN is a more robust statistic respect variations in the shape of the CCF, the mean of the SN is more sensible to changes in the asymmetry of the CCF. We suggest to use as parameter that defines the RV of the star the median of the SN, since the standard errors related to this parameter working with CCF's from real stars are on average $10\%$ smaller than the standard errors retrieved using the Normal density. To evaluate changes in the asymmetry of the CCF, we suggest to use the mean of the SN. The correlation between SN mean RV and SN FWHM and the correlation between SN mean RV and $\gamma$ (the asymmetry parameter of the SN) are much stronger than the correlations between the equivalent parameters derived using a Normal fit (RV, FWHM and BIS SPAN or the asymmetric parameters described in \citet{Figueira-2013}). The precision on the asymmetry measured by $\gamma$ is greater than the one on BIS SPAN by $\sim$15\%. Therefore when searching for rotational periods in the data, or applying Gaussian Processes to account for stellar activity signals, the SN parameters should be used.

Because of stellar activity the estimated RV's are contaminated by spurious variations. We propose to use a function that corrects from stellar activity that beyond the width and the asymmetry parameters of the CCF includes also the contrast parameter A and the fourth parameter defined as the interaction between width and the asymmetry parameters of the CCF. We found these new two parameter helpful to explain part of the spurious variations in RV's caused by stellar activity.

Finally, we also encourage the use of bootstrapping to estimate more realistic errors on the different parameters of the Normal or SN fitted to the CCF, mainly in the low SNR regime where a gain of 50\% can be reached. This takes significantly more time, but note that 100 bootstrapped dataset are enough to get a good estimation of errors.


%the RVs of the star can be calculated as the mean of the SN, whereas its asymmetry parameter $\gamma$ can be naturally used for retrieving information about the asymmetry of the CCF. Because of the natural correlation between the RVs and the asymmetry parameter, a linear combination of both $\gamma$ and the FWHM has been implemented for correcting the RVs from stellar activity, allowing us to retrieve a more precise RVs. 

%In order to get the standard errors for the estimated parameters, rather than using the common $noise$ statistic, we created a simulated procedure based on perturbing each point of the CCF according to its measurement error. Then, by getting from the original profile line $100$ simulated profiles, we run the SN fitting analyses, allowing us retrieving for all the parameter of interests an estimation for their standard errors. Concerning the standard error associated to the means of the densitys, the SN fit leads to slightly larger standard errors respect when running the Normal fit. This fact can be explained by noticing that as a simulated CCF is created according to the procedure described above, we expect slightly differences on the asymmetry of the CCF as well. The natural correlation between the RVs of the star and $\gamma$, where the mean of the SN is shifted in the direction of the asymmetry of the density, leads to larger uncertainties related to the RVs of the star. The latter consideration provides a further justification for correcting the RVs of the star from stellar activity and explains why this correction is more helpful when running the proposed analysis rather than the classic Normal fit. Concerning the standard errors related to the asymmetry parameters, when using the SN fit the uncertainties are $10 \%$ smaller than using the classic BIS SPAN. In conclusion, we showed how using the SN fit provides in an unique step a better understanding of the stellar activity through using the asymmetry parameter $\gamma$ rather than using the common statistics such as the BIS SPAN or other indicators. On top of that, when focusing on low signal-to-noise cases, the SN procedure showed how statistically significant results are retrieved with the proposed approach in terms of correlation between the asymmetry of the profile line and the RVs, whereas for the previous analyses this correlation does not result statistically significant.

%-----------------------------------------------------------------------------------------------------------------------------------------------
\section{Acknowledgements}

We are grateful to all technical and scientific collaborators of the HARPS Consortium, ESO Headquarters and ESO La Silla who have contributed with their extraordinary passion and valuable work to the success of the HARPS project.
XD is grateful to The Branco Weiss Fellowship--Society in Science for its financial support.


%-----------------------------------------------------------------------------------------------------------------------------------------------
%\section{Appendix}
\appendix
\section{Appendix} \label{appendix}

In this Appendix we present the analyses conducted on other 4 stars: HD192310, HD10700, HD215152 and finally Corot-7. 

\umberto{Add further information about the stars here presented.}\xavier{}

Table \ref{table:summaryStars} summarizes the results obtained by the SN fit and the some of the results based on the Normal fit. The results are all consistent with the conclusions derived by the analyses on Alpha Centauri b. The correlation between $\gamma$ and SN mean RV is stronger than the correlation between the BIS SPAN and RV for all the considered stars. The correlation between SN FWHM and SN mean RV is stronger than the correlation between FWHM and RV for three of the four stars.  Also for all these stars we corrected the originally estimated RV's from spurious variations in RV's caused by stellar activity, using Eq. \ref{eq:RV:correction}. Fig. \ref{fig:HD192310:correctionRV}--\ref{fig:HD215152:correctionRV} show the resulting corrected RV's. While the Normal and SN residuals, once corrected for stellar activity, are comparable for the stars $\text{HD}192310$ and $\text{HD}10700$, the results of the analyses on the star $\text{HD}215152$ (whose CCF's have lower SNR respect to the previous two analyzed stars) suggest that the residuals for the Normal are $0.054 \ms$ higher than the residuals retrieved with the SN analysis. Finally, the results of the analysis on $\text{Corot }7$, whose CCF's have lowest SNR, show that once corrected from stellar activity the residuals from the Normal fit are $0.336 \ms$ higher than the residuals retrieved with the SN analysis.

\small{
\begin{table}[!t]
\scalebox{0.45}{
\begin{tabular}{|c|c|c|c|c|c|c|c|c|}
\hline
\textbf{Star}          &\textbf{ \# CCFs}  &   \textbf{$\text{R}(\text{SN }\gamma, \text{Bis-Span})$} & \textbf{$\text{slope}(\text{SN }\gamma, \text{Bis-Span})$} &   \textbf{$\text{R}(\text{SN }\gamma, \text{SN mean RV})$} & \textbf{$\text{R}(\text{Bis-Span}, \text{RV})$} & \textbf{$\text{R}(\text{FIG BiGaussian}, \text{RV})$} & \textbf{$\text{R}(\text{SN FWHM}, \text{SN mean RV})$}  & \textbf{$\text{R}(\text{FWHM}, \text{RV})$} \\
\hline
 $\text{HD}192310  $          &    $1577$    & $0.888$ & $0.786$ & $0.669 (0.64; 0.695)$ & $0.329 (0.285; 0.373)$  & $-0.333 (-0.376; -0.289)$ & $0.666 (0.637; 0.692)$ & $0.476 (0.4367 0.514)$\\
\hline
 $\text{HD}10700 $            &    $7928$    & $0.78$ & $0.604$ & $0.322 (0.302; 0.342)$ & $-0.073 (-0.095; -0.0051)$ & $0.127 (0.105; 0.148)$ & $0.421 (0.403; 0.439)$ & $0.529 (0.513; 0.545)$ \\
\hline
 $\text{HD}215152 $          &     $273$   &  $ 0.763$ & $0.794$ & $0.571 (0.485; 0.646)$ & $-0.067 (-0.184; 0.052)$  & $0.269 (0.155; 0.376)$ & $0.210 (0.094; 0.321)$ & $-0.138 (-0.253; -0.020)$ \\
\hline
 $\text{Corot }7$     &     $173$    &  $0.814$  & $0.607$ & $0.561 (0.450; 0.656)$ & $0.092 (-0.058; 0.238)$ & $-0.335 (-0.228; -0.082)$ & $-0.709 (0.626;0.776)$ & $0.595 (0.489; 0.683)$ \\
\hline
\end{tabular}
}
\caption{Subset of notable correlations between the asymmetry parameter (and the FWHM) and the RVs for four stars: $\text{HD}192310$,  $\text{HD}10700$, $\text{HD}215152$ and $\text{Corot }7$. The complete results of the analyses of the correlations for the four stars are presented in Fig. \ref{fig:Gliese785:corrPlot}--\ref{fig:Corot7:corrPlot}.}
\label{table:summaryStars}
\end{table}
}

%HD192310
\begin{figure*}[htbp]
   \centering
\includegraphics[height = 6in]{HD19231_[4]Comparison_para.pdf} 
   \caption{Correlation between the asymmetry parameters and the RV's for $\text{HD}192310$. The last three plots show the correlation between the FWHM and the RV's for $\text{HD}192310$ using respectively the SN and the Normal fits. The p--values associated with each $R$ is statistically different from $0$.}
   \label{fig:Gliese785:corrPlot}
\end{figure*}

\begin{figure} 
   \centering
\includegraphics[height = 6in]{NEW_CORRECTIONHD19231_[3]CorrectionActivity_RadialVelocity_vs_time.pdf} 
   \caption{Set of RV's for $\text{HD}192310$  using a Normal and a SN fit before and once corrected from stellar activity. The correction is done using Eq. \ref{eq:RV:correction}. Once corrected for stellar activity, the residuals in the Normal and SN analyses are comparable.}
   \label{fig:HD192310:correctionRV}
\end{figure}

%HD10700
\begin{figure*}[htbp]
   \centering
\includegraphics[height = 6in]{HD10700_[4]Comparison_para.pdf}  
   \caption{Correlation between the asymmetry parameters and the RV's for $\text{HD}10700$. The last three plots show the correlation between the FWHM and the RV's for $\text{HD}10700$ using respectively the SN and the Normal fits. The p--values associated with each $R$ is statistically different from $0$.}
   \label{fig:Tau:corrPlot}
\end{figure*}

\begin{figure} 
   \centering
\includegraphics[height = 6in]{NEW_CORRECTIONHD10700_[3]CorrectionActivity_RadialVelocity_vs_time.pdf} 
   \caption{Set of RV's for $\text{HD}10700$  using a Normal and a SN fit before and once corrected from stellar activity. The correction is done using Eq. \ref{eq:RV:correction}. Once corrected for stellar activity, the residuals in the Normal and SN analyses are comparable.}
   \label{fig:HD10700:correctionRV}
\end{figure}

%HD215152
\begin{figure*}[htbp]
   \centering
\includegraphics[height = 6in]{HD21515_[4]Comparison_para.pdf}  
   \caption{Correlation between the asymmetry parameters and the RV's for $\text{HD}215152$. The last three plots show the correlation between the FWHM and the RV's for $\text{HD}215152$ using respectively the SN and the Normal fits. Concerning the asymmetry of the CCF, note that the p--values associated with $R$ are strongly different fro $0$ for those parameters retrieved by using the SN.}
   \label{fig:HD215152:corrPlot}
\end{figure*}

\begin{figure} 
   \centering
\includegraphics[height = 6in]{NEW_CORRECTIONHD21515_[3]CorrectionActivity_RadialVelocity_vs_time.pdf} 
   \caption{Set of RV's for $\text{HD}215152$ using a Normal and a SN fit before and once corrected from stellar activity. The correction is done using Eq. \ref{eq:RV:correction}. Once corrected for stellar activity, the residuals for the Normal are $0.054$ \ms higher than the residuals retrieved with the SN analysis.}
   \label{fig:HD215152:correctionRV}
\end{figure}

%Corot-7
\begin{figure*}[htbp]
   \centering
\includegraphics[height = 6in]{LRa01_E_[4]Comparison_para.pdf} 
   \caption{Correlation between the asymmetry parameters and the RV's for  $\text{Corot }7$. The last three plots show the correlation between the FWHM and the RV's for  $\text{Corot }7$ using respectively the SN and the Normal fits. Concerning the asymmetry of the CCF, note that the p--values associated with $R$ are strongly different fro $0$ for those parameters retrieved by using the SN.}
   \label{fig:Corot7:corrPlot}
\end{figure*}

\begin{figure} 
   \centering
\includegraphics[height = 6in]{NEW_CORRECTIONLRa01_E_[3]CorrectionActivity_RadialVelocity_vs_time.pdf} 
   \caption{Set of RV's for  $\text{Corot }7$ using a Normal and a SN fit before and once corrected from stellar activity. The correction is done using Eq. \ref{eq:RV:correction}. Once corrected for stellar activity, the residuals for the Normal are $0.336$ \ms higher than the residuals retrieved with the SN analysis.}
   \label{fig:Corot-7:correctionRV}
\end{figure}




%%%%%%
\iffalse
%\subsection{HD192310}  \label{sec:Gl785}
%
\begin{figure*}[htbp]
   \centering
\includegraphics[height = 3in]{[0]HD19231_HistogramsDiff.pdf} 
   \caption{RVs comparison for HD192310 considering a Normal and a SN fitting (using both the mean and the median).}
   \label{fig:HD192310:RV}
\end{figure*}
%
\begin{figure}[htbp]
   \centering
\includegraphics[height = 3in]{HD19231_[2]gamma_vs_bisspan.pdf} 
   \caption{Correlation between $\gamma$ and the BIS SPAN for HD192310.}
   \label{fig:Gliese785:corr.gamma}
\end{figure}
%


%\subsection{Tau Ceti}  \label{sec:Taucet}
%
\begin{figure*}[htbp]
   \centering
\includegraphics[height = 3in]{[0]TauCeti_HistogramsDiff.pdf} 
   \caption{RVs comparison for Tau Ceti considering a Normal and a SN fitting (using both the mean and the median).}
   \label{fig:Tau Ceti:RV}
\end{figure*}
%
\begin{figure}[htbp]
   \centering
\includegraphics[height = 3in]{HD10700_[2]gamma_vs_bisspan.pdf} 
   \caption{Correlation between $\gamma$ and the BIS SPAN for Tau Ceti.}
   \label{fig:Tau:corr.gamma}
\end{figure}
%


%\subsection{HD215152}  \label{sec:HD215152}
%
\begin{figure*}[htbp]
   \centering
\includegraphics[height = 3in]{[0]HD21515_HistogramsDiff.pdf} 
   \caption{RVs comparison for HD215152 considering a Normal and a SN fitting (using both the mean and the median).}
   \label{fig:HD215152:RV}
\end{figure*}
%
\begin{figure}[htbp]
   \centering
\includegraphics[height = 3in]{HD21515_[2]gamma_vs_bisspan.pdf} 
   \caption{Correlation between $\gamma$ and the BIS SPAN for HD 215152.}
   \label{fig:HD215152:corr.gamma}
\end{figure}
%


%\subsection{Corot-7}  \label{sec:Corot7}
%
\begin{figure*}[htbp]
   \centering
\includegraphics[height = 3in]{[0]LRa01_E_HistogramsDiff.pdf} 
   \caption{RVs comparison for Corot-7 considering a Normal and a SN fitting (using both the mean and the median).}
   \label{fig:corot7:RV}
\end{figure*}
%
\begin{figure}[htbp]
   \centering
\includegraphics[height = 3in]{LRa01_E_[2]gamma_vs_bisspan.pdf} 
   \caption{Correlation between $\gamma$ and the BIS SPAN for Corot-7.}
   \label{fig:Corot7:corr.gamma}
\end{figure}
%
\fi
%

%%-----------------------------------------------------------------------------------------------------------------------------------------------
\bibliographystyle{aa}
%\bibliography{dumusque_bibliography}
\bibliography{mybib-SNCCF}

%\begin{appendix}
%\end{appendix}

\end{document}
